% Options for packages loaded elsewhere
\PassOptionsToPackage{unicode}{hyperref}
\PassOptionsToPackage{hyphens}{url}
%
\documentclass[
]{book}
\usepackage{amsmath,amssymb}
\usepackage{iftex}
\ifPDFTeX
  \usepackage[T1]{fontenc}
  \usepackage[utf8]{inputenc}
  \usepackage{textcomp} % provide euro and other symbols
\else % if luatex or xetex
  \usepackage{unicode-math} % this also loads fontspec
  \defaultfontfeatures{Scale=MatchLowercase}
  \defaultfontfeatures[\rmfamily]{Ligatures=TeX,Scale=1}
\fi
\usepackage{lmodern}
\ifPDFTeX\else
  % xetex/luatex font selection
\fi
% Use upquote if available, for straight quotes in verbatim environments
\IfFileExists{upquote.sty}{\usepackage{upquote}}{}
\IfFileExists{microtype.sty}{% use microtype if available
  \usepackage[]{microtype}
  \UseMicrotypeSet[protrusion]{basicmath} % disable protrusion for tt fonts
}{}
\makeatletter
\@ifundefined{KOMAClassName}{% if non-KOMA class
  \IfFileExists{parskip.sty}{%
    \usepackage{parskip}
  }{% else
    \setlength{\parindent}{0pt}
    \setlength{\parskip}{6pt plus 2pt minus 1pt}}
}{% if KOMA class
  \KOMAoptions{parskip=half}}
\makeatother
\usepackage{xcolor}
\usepackage{longtable,booktabs,array}
\usepackage{calc} % for calculating minipage widths
% Correct order of tables after \paragraph or \subparagraph
\usepackage{etoolbox}
\makeatletter
\patchcmd\longtable{\par}{\if@noskipsec\mbox{}\fi\par}{}{}
\makeatother
% Allow footnotes in longtable head/foot
\IfFileExists{footnotehyper.sty}{\usepackage{footnotehyper}}{\usepackage{footnote}}
\makesavenoteenv{longtable}
\usepackage{graphicx}
\makeatletter
\def\maxwidth{\ifdim\Gin@nat@width>\linewidth\linewidth\else\Gin@nat@width\fi}
\def\maxheight{\ifdim\Gin@nat@height>\textheight\textheight\else\Gin@nat@height\fi}
\makeatother
% Scale images if necessary, so that they will not overflow the page
% margins by default, and it is still possible to overwrite the defaults
% using explicit options in \includegraphics[width, height, ...]{}
\setkeys{Gin}{width=\maxwidth,height=\maxheight,keepaspectratio}
% Set default figure placement to htbp
\makeatletter
\def\fps@figure{htbp}
\makeatother
\setlength{\emergencystretch}{3em} % prevent overfull lines
\providecommand{\tightlist}{%
  \setlength{\itemsep}{0pt}\setlength{\parskip}{0pt}}
\setcounter{secnumdepth}{5}
\usepackage{booktabs}
\usepackage{amsthm}
\usepackage[pagebackref=false,breaklinks=true,letterpaper=true,colorlinks,bookmarks=false]{hyperref}
\usepackage{tcolorbox}
\usepackage{color}
\usepackage{framed}
\setlength{\fboxsep}{.8em}
\makeatletter
\def\thm@space@setup{%
  \thm@preskip=8pt plus 2pt minus 4pt
  \thm@postskip=\thm@preskip
}
\makeatother


\newtcolorbox{examplebox}{
  colback=green,
  colframe=orange,
  coltext=black,
  boxsep=5pt,
  arc=4pt}
\ifLuaTeX
  \usepackage{selnolig}  % disable illegal ligatures
\fi
\usepackage[]{natbib}
\bibliographystyle{apalike}
\IfFileExists{bookmark.sty}{\usepackage{bookmark}}{\usepackage{hyperref}}
\IfFileExists{xurl.sty}{\usepackage{xurl}}{} % add URL line breaks if available
\urlstyle{same}
\hypersetup{
  pdftitle={Optimal Control and Estimation},
  pdfauthor={Heng Yang},
  hidelinks,
  pdfcreator={LaTeX via pandoc}}

\title{Optimal Control and Estimation}
\author{Heng Yang}
\date{2023-06-29}

\usepackage{amsthm}
\newtheorem{theorem}{Theorem}[chapter]
\newtheorem{lemma}{Lemma}[chapter]
\newtheorem{corollary}{Corollary}[chapter]
\newtheorem{proposition}{Proposition}[chapter]
\newtheorem{conjecture}{Conjecture}[chapter]
\theoremstyle{definition}
\newtheorem{definition}{Definition}[chapter]
\theoremstyle{definition}
\newtheorem{example}{Example}[chapter]
\theoremstyle{definition}
\newtheorem{exercise}{Exercise}[chapter]
\theoremstyle{definition}
\newtheorem{hypothesis}{Hypothesis}[chapter]
\theoremstyle{remark}
\newtheorem*{remark}{Remark}
\newtheorem*{solution}{Solution}
\begin{document}
\maketitle

{
\setcounter{tocdepth}{1}
\tableofcontents
}
\hypertarget{preface}{%
\chapter*{Preface}\label{preface}}
\addcontentsline{toc}{chapter}{Preface}

This is the textbook for Harvard ES/AM 158: Introduction to Optimal Control and Estimation. Information about the offerings of the class is listed below.

\hypertarget{fall}{%
\subsubsection*{2023 Fall}\label{fall}}
\addcontentsline{toc}{subsubsection}{2023 Fall}

\textbf{Time}: Mon/Wed 2:15 - 3:30pm

\textbf{Location}: Science and Engineering Complex, Room TBD

\textbf{Instructor}: \href{https://hankyang.seas.harvard.edu/}{Heng Yang}

\textbf{Teaching Fellow}: \href{https://scholar.harvard.edu/weiyuli/home}{Weiyu Li}

\href{https://docs.google.com/document/d/1q8_jB5dLx9jHOBi3DQ48Vv2E243ocGCGm_H0mJuOojM/edit?usp=sharing}{\textbf{Syllabus}}

\hypertarget{acknowledgment}{%
\subsubsection*{Acknowledgment}\label{acknowledgment}}
\addcontentsline{toc}{subsubsection}{Acknowledgment}

\hypertarget{formulation}{%
\chapter{The Optimal Control Formulation}\label{formulation}}

\hypertarget{the-basic-problem}{%
\section{The Basic Problem}\label{the-basic-problem}}

Consider a discrete-time dynamical system
\begin{equation}
x_{k+1} = f_k (x_k, u_k, w_k), \quad k =0,1,\dots,N-1
\label{eq:discrete-time-dynamics}
\end{equation}
where

\begin{itemize}
\item
  \(x_k \in \mathbb{X} \subseteq \mathbb{R}^n\) is the \emph{state} of the system,
\item
  \(u_k \in \mathbb{U} \subseteq \mathbb{R}^m\) is the \emph{control} we wish to design,
\item
  \(w_k \in \mathbb{W} \subseteq \mathbb{R}^p\) a random \emph{disturbance} or noise (e.g., due to unmodelled dynamics) which is described by a probability distribution \(P_k(\cdot \mid x_k, u_k)\) that may depend on \(x_k\) and \(u_k\) but not on prior disturbances \(w_0,\dots,w_{k-1}\),
\item
  \(k\) indexes the discrete time,
\item
  \(N\) denotes the horizon,
\item
  \(f_k\) models the transition function of the system (typically \(f_k \equiv f\) is time-invariant, especially for robotics systems; we use \(f_k\) here to keep full generality).
\end{itemize}

\begin{remark}[Deterministic v.s. Stochastic]
When \(w_k \equiv 0\) for all \(k\), we say the system \eqref{eq:discrete-time-dynamics} is \emph{deterministic}; otherwise we say the system is \emph{stochastic}. In the following we will deal with the stochastic case, but most of the methodology should carry over to the deterministic setup.
\end{remark}

We consider the class of \emph{controllers} (also called \emph{policies}) that consist of a sequence of functions
\[
\pi = \{ \mu_0,\dots,\mu_{N-1} \},
\]
where \(\mu_k (x_k) \in \mathbb{U}\) for all \(x_k\), i.e., \(\mu_k\) is a \emph{feedback} controller that maps the state to an admissible control. Given an initial state \(x_0\) and an admissible policy \(\pi\), the state \emph{trajectory} of the system is a sequence of random variables that evolve according to
\begin{equation}
x_{k+1} = f_k(x_k,\mu_k(x_k),w_k), \quad k=0,\dots,N-1
\label{eq:closed-loop-state-trajectory}
\end{equation}
where the randomness comes from the disturbance \(w_k\).

We assume the state-control trajectory \(\{u_k\}_{k=0}^{N-1}\) and \(\{x_k \}_{k=0}^{N}\) induce an \emph{additive cost}
\begin{equation}
g_N(x_N) + \sum_{k=0}^{N-1} g_k(x_k,u_k)
\label{eq:additive-cost}
\end{equation}
where \(g_k,k=0,\dots,N\) are some user-designed functions.

With \eqref{eq:closed-loop-state-trajectory} and \eqref{eq:additive-cost}, for any admissible policy \(\pi\), we denote its induced \emph{expected cost} with initial state \(x_0\) as
\begin{equation}
J_\pi (x_0) = \mathbb{E} \left\{ g_N(x_N) + \sum_{k=0}^{N-1} g_k (x_k, \mu_k(x_k))  \right\},
\label{eq:expected-cost}
\end{equation}
where the expectation is taken over the randomness of \(w_k\).

\begin{definition}[Discrete-time, Finite-horizon Optimal Control]
\protect\hypertarget{def:basicproblem}{}\label{def:basicproblem}Find the best admissible controller that minimizes the expected cost in \eqref{eq:expected-cost}
\begin{equation}
\pi^\star \in \arg\min_{\pi \in \Pi} J_\pi(x_0),
\end{equation}
where \(\Pi\) is the set of all admissible controllers.
The cost attained by the optimal controller, i.e., \(J^\star = J_{\pi^\star}(x_0)\) is called the optimal \emph{cost-to-go}, or the optimal \emph{value function}.
\end{definition}

\begin{remark}[Open-loop v.s. Closed-loop]
An important feature of the basic problem in Definition \ref{def:basicproblem} is that the problem seeks \emph{feedback policies}, instead of numerical values of the controls, i.e., \(u_k = \mu_k(x_k)\) is in general a function of the state \(x_k\). In other words, the controls are executed sequentially, one at a time after observing the state at each time. This is called closed-loop control, and is in general better than open-loop control
\[
\min_{u_0,\dots,u_{N-1}} \mathbb{E} \left\{ g_N(x_N) + \sum_{k=0}^{N-1} g_k (x_k, u_k)  \right\}
\]
where all the controls are planned at \(k=0\). Intuitively, a closed-loop policy is able to utilize the extra information received at each timestep (i.e., it observes \(x_{k+1}\) and hence also observes the disturbance \(w_k\)) to obtain a lower cost than an open-loop controller. Example 1.2.1 in \citep{bertsekas12book-dpocI} gives a concrete application where a closed-loop policy attains a lower cost than an open-loop policy.

In deterministic control (i.e., when \(w_k \equiv 0,\forall k\)), however, a closed-loop policy has no advantage over an open-loop controller. This is obvious because at \(k=0\), even the open-loop controller predicts perfectly the consequences of all its actions and there is no extra information to be observed at later time steps. In fact, even in stochastic problems, a closed-loop policy may not be advantageous, see Exercise 1.27 in \citep{bertsekas12book-dpocI}.
\end{remark}

\hypertarget{dynamic-programming-and-principle-of-optimality}{%
\section{Dynamic Programming and Principle of Optimality}\label{dynamic-programming-and-principle-of-optimality}}

We now introduce a general and powerful algorithm, namely \emph{dynamic programming} (DP), for solving the optimal control problem \ref{def:basicproblem}. The DP algorithm builds upon a quite simple intuition called the \emph{Bellman principle of optimality}.

\begin{theorem}[Bellman Principle of Optimality]
\protect\hypertarget{thm:bellmanoptimality}{}\label{thm:bellmanoptimality}Let \(\pi^\star = \{ \mu_0^\star,\mu_1^\star,\dots,\mu_{N-1}^\star \}\) be an optimal policy for the optimal control problem \ref{def:basicproblem}. Assume that when using \(\pi^\star\), a given state \(x_i\) occurs at timestep \(i\) with positive probability (i.e., \(x_i\) is reachable at time \(i\)).

Now consider the following subproblem where we are at \(x_i\) at time \(i\) and wish to minimize the cost-to-go from time \(i\) to time \(N\)
\[
\min_{\mu_i,\dots,\mu_{N-1}} \mathbb{E} \left\{ g_N(x_N) + \sum_{k=i}^{N-1} g_k (x_k, \mu_k(x_k)) \right\}.
\]
Then the truncated policy \(\{\mu^\star_i,\mu^\star_{i+1},\dots, \mu^\star_{N-1}\}\) must be optimal for the subproblem.
\end{theorem}

Theorem \ref{thm:bellmanoptimality} can be proved intuitively by contradiction: if the truncated policy \(\{\mu^\star_i,\mu^\star_{i+1},\dots, \mu^\star_{N-1}\}\) is not optimal for the subproblem, say there exists a different policy \(\{\mu_i',\mu_{i+1}',\dots, \mu_{N-1}'\}\) that attains a lower cost for the subproblem starting at \(x_i\) at time \(i\). Then the combined policy \(\{\mu_0^\star,\dots,\mu^\star_{i-1},\mu_i',\dots,\mu_{N-1}'\}\) must attain a lower cost for the original optimal control problem \ref{def:basicproblem} due to the additive cost structure, contradicting the optimality of \(\pi^\star\).

The Bellman principle of optimality is more than just a principle, it is also an algorithm. It suggests that, to build an optimal policy, one can start by solving the last-stage subproblem to obtain \(\{\mu^\star_{N-1} \}\), and then proceed to solve the subproblem containing the last two stages to obtain \(\{ \mu^\star_{N-2},\mu^\star_{N-1} \}\). The recursion continues until optimal policies at all stages are computed. The following theorem formalizes this concept.

\begin{theorem}[Dynamic Programming]
\protect\hypertarget{thm:dynamicprogramming}{}\label{thm:dynamicprogramming}The optimal value function \(J^\star(x_0)\) of the optimal control problem \ref{def:basicproblem} (starting from any given initial condition \(x_0\)) is equal to \(J_0(x_0)\), which can be computed backwards and recursively as
\begin{align}
J_N(X_N) &= g_N(x_N) \\
J_k(x_k) &= \min_{u_k \in \mathbb{U}} \displaystyle \mathbb{E}_{w_k \sim P_k(\cdot \mid x_k, u_k)} \displaystyle \left\{ g_k(x_k,u_k) + J_{k+1}(f_k(x_k,u_k,w_k) ) \right\}, \ k=N-1,\dots,1,0.
\label{eq:dpbackwardrecursion}
\end{align}
Moreover, if \(u_k^\star = \mu_k^\star(x_k)\) is a minimizer of \eqref{eq:dpbackwardrecursion} for every \(x_k\), then the policy \(\pi^\star = \{\mu_0^\star,\dots,\mu_{N-1}^\star \}\) is optimal.
\end{theorem}

\begin{proof}
For any admissible policy \(\pi = \{ \mu_0,\dots,\mu_{N-1} \}\), denote \(\pi^k = \{ \mu_k,\dots,\mu_{N-1} \}\) the last-\((N-k)\)-stage truncated policy. Consider the subproblem consisting of the last \(N-k\) stages starting from \(x_k\), and let \(J^\star_k(x_k)\) be its optimal cost-to-go. Mathematically, this is
\begin{equation}
J^\star_{k}(x_k) = \min_{\pi^k} \mathbb{E}_{w_k,\dots,w_{N-1}} \left\{ g_N(x_N) + \sum_{i=k}^{N-1} g_i (x_i,\mu_i(x_i)) \right\}, \quad k=0,1,\dots,N-1.
\label{eq:dptheoremdefineJkstar}
\end{equation}
We define \(J^\star_N(x_N) = g(x_N)\) for \(k=N\).

Our goal is to prove the \(J_k(x_k)\) computed by dynamic programming from \eqref{eq:dpbackwardrecursion} is equal to \(J^\star_k (x_k)\) for all \(k=0,\dots,N\). We will prove this by induction.

Firstly, we already have \(J^\star_N(x_N) = J_N(x_N) = g(x_N)\), so \(k=N\) holds automatically.

Now we assume \(J^\star_{k+1}(x_{k+1}) = J_{k+1}(x_{k+1})\) for all \(x_{k+1}\), and we wish to induce \(J^\star_{k}(x_{k}) = J_{k}(x_{k})\). To show this, we write
\begin{align}
\hspace{-16mm} J^\star_{k}(x_k) &= \min_{\pi^k} \mathbb{E}_{w_k,\dots,w_{N-1}} \left\{ g_N(x_N) + \sum_{i=k}^{N-1} g_i (x_i,\mu_i(x_i)) \right\} \label{eq:dpproof-1}\\
&= \min_{\mu_k,\pi^{k+1}} \mathbb{E}_{w_k,\dots,w_{N-1}} \left\{ g_k(x_k,\mu_k(x_k)) + g_N(x_N) + \sum_{i=k+1}^{N-1} g_i(x_i,\mu_i(x_i))  \right\}
\label{eq:dpproof-2}\\
&= \min_{\mu_k} \left[ \min_{\pi^{k+1}} \mathbb{E}_{w_k,\dots,w_{N-1}} \left\{ g_k(x_k,\mu_k(x_k)) + g_N(x_N) + \sum_{i=k+1}^{N-1} g_i(x_i,\mu_i(x_i))  \right\}\right] \label{eq:dpproof-3}\\
&= \min_{\mu_k} \mathbb{E}_{w_k} \left\{ g_k(x_k,\mu_k(x_k)) + \min_{\pi^{k+1}} \left[ \mathbb{E}_{w_{k+1},\dots,w_{N-1}} \left\{ g_N(x_N) + \sum_{i=k+1}^{N-1} g_i(x_i,\mu_i(x_i))  \right\}  \right]    \right\} \label{eq:dpproof-4}\\
&= \min_{\mu_k} \mathbb{E}_{w_k} \left\{ g_k(x_k,\mu_k(x_k)) + J^\star_{k+1}(f_k(x_k,\mu_k(x_k),w_k)) \right\} \label{eq:dpproof-5}\\
&= \min_{\mu_k} \mathbb{E}_{w_k} \left\{ g_k(x_k,\mu_k(x_k)) + J_{k+1}(f_k(x_k,\mu_k(x_k),w_k)) \right\} \label{eq:dpproof-6}\\
&= \min_{u_k \in \mathbb{U}} \mathbb{E}_{w_k} \left\{ g_k(x_k,\mu_k(x_k)) + J_{k+1}(f_k(x_k,\mu_k(x_k),w_k)) \right\} \label{eq:dpproof-7}\\
&= J_k(x_k), \label{eq:dpproof-8}
\end{align}
where \eqref{eq:dpproof-1} follows from definition \eqref{eq:dptheoremdefineJkstar}; \eqref{eq:dpproof-2} expands \(\pi^k = \{ \mu_k, \pi^{k+1}\}\) and \(\sum_{i=k}^{N-1} g_i = g_k + \sum_{i=k+1}^{N-1}\); \eqref{eq:dpproof-3} writes the joint minimization over \(\mu_k\) and \(\pi^{k+1}\) as equivalently first minimizing over \(\pi^{k+1}\) and then minimizing over \(\mu_k\); \eqref{eq:dpproof-4} is the key step and holds because \(g_k\) and \(w_k\) depend only on \(\mu_k\) but not on \(\pi^{k+1}\); \eqref{eq:dpproof-5} follows again from definition \eqref{eq:dptheoremdefineJkstar} with \(k\) replaced by \(k+1\); \eqref{eq:dpproof-6} results from the induction assumption; \eqref{eq:dpproof-7} clearly holds because any \(\mu_k(x_k)\) belongs to \(\mathbb{U}\) and any element in \(\mathbb{U}\) can be chosen by a feedback controller \(\mu_k\); and lastly \eqref{eq:dpproof-8} follows from the dynamic programming algorithm \eqref{eq:dpbackwardrecursion}.

By induction, this shows that \(J^\star_k(x_k) = J_k(x_k)\) for all \(k=0,\dots,N\).
\end{proof}

The careful reader, especially from a robotics background, may soon become disappointed when seeing the DP algorithm \eqref{eq:dpbackwardrecursion} because it is rather conceptual than practical. To see this, we only need to run DP for \(k=N-1\):
\begin{equation}
J_{N-1}(x_{N-1}) = \min_{u_{N-1} \in \mathbb{U}} \mathbb{E}_{w_{N-1}} \left\{ g_{N-1}(x_{N-1},u_{N-1}) + J_N(f_{N-1}(x_{N-1},u_{N-1},w_{N-1})) \right\}.
\label{eq:dptryN-1}
\end{equation}

Two challenges immediately show up:

\begin{itemize}
\item
  How to perform the minimization over \(u_{N-1}\) when \(\mathbb{U}\) is a continuous constraint set? Even if we assume \(g_{N-1}\) is convex\footnote{You may want to read Appendix \ref{appconvex} if this is your first time seeing ``convex'' things.} in \(u_{N-1}\), \(J_N\) is convex in \(x_{N}\), and the dynamics \(f_{N-1}\) is also convex in \(u_{N-1}\) (so that the optimization \eqref{eq:dptryN-1} is convex), we may be able to solve the minimization \emph{numerically} for each \(x_{N-1}\) using a convex optimization solver, but rarely will we be able to find an analytical policy \(\mu_{N-1}^\star\) such that \(u_{N-1}^\star = \mu_{N-1}^\star (x_{N-1})\) for every \(x_{N-1}\) (i.e., the optimal policy \(\mu_{N-1}^\star\) is implict but not explict).
\item
  Suppose we can find an anlytical optimal policy \(\mu_{N-1}^\star\), say \(\mu_{N-1}^\star = K x_{N-1}\) a linear policy, how will plugging \(\mu_{N-1}^\star\) into \eqref{eq:dptryN-1} affect the complexity of \(J_{N-1}(x_{N-1})\)? One can see that even if \(\mu_{N-1}^\star\) is linear in \(x_{N-1}\), \(J_{N-1}\) may be highly nonlinear in \(x_{N-1}\) due to the composition with \(g_{N-1}\), \(f_{N-1}\) and \(J_N\). If \(J_{N-1}(x_{N-1})\) becomes too complex, then clearly it becomes more challenging to perform \eqref{eq:dptryN-1} for the next step \(k=N-2\).
\end{itemize}

Due to these challenges, only in a very limited amount of cases will we be able to perform \emph{exact dynamic programming}. For example, when the state space \(\mathbb{X}\) and control space \(\mathbb{U}\) are discrete, we can design efficient algorithms for exact DP. For another example, when the dynamics \(f_k\) is linear and the cost \(g_k\) is quadratic, we will also be able to compute \(J_k(x_k)\) in closed form (though this sounds a bit surprising!). We will study these problems in more details in Chapter \ref{exactdp}.

For general optimal control problems with continuous state space and control space (and most problems we care about in robotics), unfortunately, we will have to resort to \emph{approximate dynamic programming}, basically variations of the DP algorithm \eqref{eq:dpbackwardrecursion} where approximate value functions \(J_k(x_k)\) and/or control policies \(\mu_k(x_k)\) are used (e.g., with neural networks and machine learning).\footnote{Another possible solution is to discretize continuous states and controls. However, when the dimension of state and control is high, discretization becomes too expensive in terms of memory and computational complexity.} We will introduce several popular approximation schemes in Chapter \ref{approximatedp}. We will see that, although exact DP is not possible anymore, the Bellman principle of optimality still remains one of the most important guidelines for designing approximation algorithms. Efficient algorithms for approximate dynamic programming, preferrably with performance guarantees, still remain an active area of research.

\hypertarget{exactdp}{%
\chapter{Exact Dynamic Programming}\label{exactdp}}

\hypertarget{lqr}{%
\section{Linear Quadratic Regulator}\label{lqr}}

\hypertarget{approximatedp}{%
\chapter{Approximate Dynamic Programming}\label{approximatedp}}

\hypertarget{stability}{%
\chapter{Stability Analysis}\label{stability}}

\begin{lemma}[Barbalat's Lemma]
\protect\hypertarget{lem:Barbalat}{}\label{lem:Barbalat}Let \(f(t)\) be differentiable, if

\begin{itemize}
\item
  \(\lim_{t \rightarrow \infty} f(t)\) is finite, and
\item
  \(\dot{f}(t)\) is uniformly continuous,\footnote{A sufficient condition for this to hold is that \(\ddot{f}\) exists and is bounded.}
\end{itemize}

then
\[
\lim_{t \rightarrow \infty} \dot{f}(t) = 0.
\]
\end{lemma}

\begin{theorem}[Barbalat's Stability Certificate]
\protect\hypertarget{thm:BarbalatStability}{}\label{thm:BarbalatStability}If a scalar function \(V(x,t)\) satisfies

\begin{itemize}
\item
  \(V(x,t)\) is lower bounded,
\item
  \(\dot{V}(x,t)\) is negative semidefinite
\item
  \(\dot{V}(x,t)\) is uniformly continuous
\end{itemize}

then \(\dot{V}(x,t) \rightarrow 0\) as \(t \rightarrow \infty\).
\end{theorem}

\begin{proof}
\(V(x,t)\) is lower bounded and \(\dot{V}\) is negative semidefinite implies the limit of \(V\) as \(t \rightarrow \infty\) is finite (note that \(V(x,t) \leq V(x(0),0)\)). Then the theorem clearly follows from Barbalat's Lemma \ref{lem:Barbalat}.
\end{proof}

\hypertarget{output-feedback}{%
\chapter{Output Feedback}\label{output-feedback}}

Consider a continuous-time dynamical system
\begin{equation}
\begin{split}
\dot{x} &= f(x,u)  \\
y &= h(x,u)
\end{split} 
\label{eq:output-feedback-system}
\end{equation}
where \(x(t) \in \mathbb{X} \subseteq \mathbb{R}^n\) the state of the system, \(u(t) \in \mathbb{U} \subseteq \mathbb{R}^m\) the control (or input), \(y(t) \in \mathbb{Y} \subseteq \mathbb{R}^{d}\) the output (i.e., measurement) of the state and control, and \(f,g\) the evolution and measurement functions (which are sufficiently smooth).

\hypertarget{state-observer}{%
\section{State Observer}\label{state-observer}}

For the system \eqref{eq:output-feedback-system}, let us denote

\begin{itemize}
\item
  \(X(x_0,t_0;t;u)\) the solution at time \(t\) with input \(u\) and initial condition \(x_0\) at time \(t_0\); when \(t_0 = 0\), we write \(X(x_0;t;u)\)
\item
  \(Y(x_0,t_0;t;u)\) the output at time \(t\) with input \(u\) and initial condition \(x_0\) at time \(t_0\), i.e., \(Y(x_0,t_0;t;u) = h(X(x_0,t_0;t;u), u(t))\); when \(t_0 = 0\), we write \(y_{x_0,u}(t)\);
\item
  \(\mathcal{X}_0\) a subset of \(\mathbb{X}\) containing the initial conditions we consider; for any \(x_0 \in \mathcal{X}_0\), we write \(\sigma^+_{\mathcal{X}}(x_0;u)\) the maximal time of existence of \(X(x_0,\cdot;t;u)\) in a set \(\mathcal{X}\)
\item
  \(\mathcal{U}\) the set of all sufficiently many times differentiable inputs \(u: [0,+\infty) \rightarrow \mathbb{U}\).
\end{itemize}

The problem of state observation is to produce an estimated state \(\hat{x}(t)\) of the true state \(X(x_0,t_0;t;u)\) based on knowledge about the system \eqref{eq:output-feedback-system} and information about the history of inputs \(u_{[0,t]}\) and outputs \(y_{[0,t]}\), so that \(\hat{x}(t)\) asymptotically converges to \(X(x_0,t_0;t;u)\), for any initial condition \(x_0 \in \mathcal{X}_0\) and any input \(u \in \mathcal{U}\).

There are multiple ways for solving the problem of state observation (see e.g., \citep{bernard19book-observer}, \citep{bernard22arc-observer}). Here we are particularly interested in the approach using a \emph{state observer}, i.e., a dynamical system whose \emph{internal state} evolves according to the history of inputs and outputs, from which a state estimation can be reconstructed that guarantees asymptotic convergence to the true state. We formalize this concept below.

\begin{definition}[State Observer]
\protect\hypertarget{def:stateobserver}{}\label{def:stateobserver}A state observer for system \eqref{eq:output-feedback-system} is a couple \((\mathcal{F},\mathcal{T})\) such that

\begin{enumerate}
\def\labelenumi{\arabic{enumi}.}
\item
  \(\mathcal{F}: \mathbb{R}^{l} \times \mathbb{R}^{m} \times \mathbb{R}^d \rightarrow \mathbb{R}^l\) is continuous
\item
  \(\mathcal{T}\) is a family of continuous functions indexed by \(u \in \mathcal{U}\) where each \(\mathcal{T}_u: \mathbb{R}^l \times [0,+\infty) \rightarrow \mathbb{R}^n\) respects the causality condition
  \[
  \forall \tilde{u}: [0,+\infty) \rightarrow \mathbb{R}^m,\forall t \in [0,+\infty), u_{[0,t]} = \tilde{u}_{[0,t]} \Rightarrow  \mathcal{F}_u (\cdot,t) = \mathcal{F}_{\tilde{u}}(\cdot,t).
  \]
\item
  For any \(u \in \mathcal{U}\), any \(z_0 \in \mathbb{R}^l\), and any \(x_0 \in \mathcal{X}_0\) such that \(\sigma^+_{\mathbb{X}}(x_0;u) = +\infty\), any solution \(Z(z_0;t;u,y_{x_0,u})\)\footnote{We say ``any solution'' because there may be several solutions to the observer \eqref{eq:observer-definition-1} due to \(\mathcal{F}\) only being continuous. This is not a problem as long as any such solution satisfies the required convergence property.} to
  \begin{equation}
  \dot{z} = \mathcal{F}(z,u,y_{x_0,u})
  \label{eq:observer-definition-1}
  \end{equation}
  initialized at \(z_0\) at time \(0\) with input \(u\) and \(y_{x_0,u}\) exists on \([0,+\infty)\) and satisfies
  \begin{equation}
  \lim_{t \rightarrow \infty} \Vert \hat{X}(x_0,z_0;t;u) - X(x_0;t;u) \Vert = 0,
  \label{eq:observer-definition-2}
  \end{equation}
  with
  \begin{equation}
  \hat{X}(x_0,z_0;t;u) = \mathcal{T}_u(Z(z_0;t;u,y_{x_0,u}),t).
  \label{eq:observer-definition-3}
  \end{equation}
  In words, (i) the state observer maintains an internal state (or latent state) \(z \in \mathbb{R}^l\) that evolves according to the latent dynamics \(\mathcal{F}\) in \eqref{eq:observer-definition-1}, where \(u\) and \(y_{x_0,u}\) are inputs; (ii) an estimated state can be reconstructed from the internal state using \(\mathcal{T}_u\) as in \eqref{eq:observer-definition-3}; and (iii) the error between the estimated state and the true state (defined by a proper distance function \(\Vert \cdot \Vert\) on \(\mathbb{X}\)) converges to zero.
\end{enumerate}

If \(\mathcal{T}_u\) is the same for any \(u \in \mathcal{U}\) and is also time independent, then we say \(\mathcal{T}\) is \emph{stationary}.\footnote{The time dependence of \(\mathcal{T}_u\) enables us to cover the case where the knowledge of the \(u\) and \(y_{x_0,u}\) is used to construct the estimate from the observer state. In particular, using the output sometimes can reduce the dimension of the observer state (and thus alleviate the computations), thus obtaining a reduced-order observer. For example, see \citep{karagiannis05cdc-nonlinear} and \citep{astolfi03tac-immersion}.} In this case, we can simply write the observer \eqref{eq:observer-definition-1} and \eqref{eq:observer-definition-3} as
\begin{equation}
\begin{split}
\dot{z} &= \mathcal{F}(z,u,y) \\
\hat{x} &= \mathcal{T}(z).
\end{split}
\label{eq:observer-definition-simple}
\end{equation}

If \(\hat{x}\) can be read off directly from \(z\), then we say the observer \eqref{eq:observer-definition-simple} is \emph{in the given coordinates}. A special case of this is when \(\hat{x} = z\), i.e., the internal state of the observer is the same as the system state.
\end{definition}

\hypertarget{general-design-strategy}{%
\subsection{General Design Strategy}\label{general-design-strategy}}

\begin{theorem}[Meta Observer]
\protect\hypertarget{thm:observerdesignmeta}{}\label{thm:observerdesignmeta}Let \(F: \mathbb{R}^p \times \mathbb{R}^m \times \mathbb{R}^d \rightarrow \mathbb{R}^p\), \(H: \mathbb{R}^p \times \mathbb{R}^m \rightarrow \mathbb{R}^d\) and \(\mathcal{F}: \mathbb{R}^p \times \mathbb{R}^m \times \mathbb{R}^d \rightarrow \mathbb{R}^p\) be continuous functions such that
\begin{equation}
\dot{\hat{\xi}} = \mathcal{F}(\hat{\xi}, u, \tilde{y})
\label{eq:meta-observer-zeta-hat}
\end{equation}
is an observer for
\begin{equation}
\dot{\xi} = F(\xi,u,H(\xi,u)), \quad \tilde{y} = H(\xi,u),
\label{eq:meta-observer-zeta}
\end{equation}
i.e., for any \(\xi_0,\hat{\xi}_0 \in \mathbb{R}^p\) and any \(u \in \mathcal{U}\), the solution of the observer \eqref{eq:meta-observer-zeta-hat},
denoted by \(\hat{\Xi}(\hat{\xi}_0;t;u;\tilde{y}_{\xi_0,u})\), and the solution of the true system \eqref{eq:meta-observer-zeta}, denoted by \(\Xi(\xi_0;t;u)\), satisfy
\begin{equation}
\lim_{t \rightarrow \infty} \Vert \hat{\Xi}(\hat{\xi}_0;t;u;\tilde{y}_{\xi_0,u}) - \Xi(\xi_0;t;u) \Vert = 0.
\label{eq:meta-observer-zeta-converge}
\end{equation}
Note that the observer \eqref{eq:meta-observer-zeta-hat} is stationary and in the given coordinates for system \eqref{eq:meta-observer-zeta}. Indeed the internal state of the observer is the same as the system state.

Now suppose for any \(u \in \mathcal{U}\), there exists a continuous function (i.e., coordinate transformation) \(T_u: \mathbb{R}^n \times \mathbb{R} \rightarrow \mathbb{R}^p\) and a subset \(\mathcal{X}\) of \(\mathbb{X}\) such that

\begin{enumerate}
\def\labelenumi{\arabic{enumi}.}
\item
  For any \(x_0 \in \mathcal{X}_0\) such that \(\sigma^+_{\mathbb{X}}(x_0;u) = + \infty\), \(X(x_0;\cdot;u)\) remains in \(\mathcal{X}\)
\item
  There exists a concave \(\mathcal{K}\)\footnote{A function \(\rho: \mathbb{R}_+ \rightarrow \mathbb{R}_+\) is a \(\mathcal{K}\) function if \(\rho(0) = 0\), \(\rho\) is continuous, and \(\rho\) is increasing.} function \(\rho\) and a positive number \(\bar{t}\) such that
  \[
  \Vert x_a - x_b \Vert \leq \rho (| T_u(x_a,t) - T_u(x_b,t) |), \quad \forall x_a,x_b \in \mathcal{X}, t \geq \bar{t},
  \]
  i.e., \(x \mapsto T_u(x,t)\) becomes injective on \(\mathcal{X}\),\footnote{An injective function is a function \(f\) that maps distinct elements of its domain to distinct elements. That is, \(f(x_a) = f(x_b)\) implies \(x_a = x_b\), or equivalently, \(x_a \neq x_b\) implies \(f(x_a) \neq f(x_b)\).} uniformly in time and space, after a certain time \(\bar{t}\).
\item
  \(T_u\) transforms the system \eqref{eq:output-feedback-system} into the system \eqref{eq:meta-observer-zeta}, i.e., for all \(x \in \mathcal{X}\) and all \(t \geq 0\), we have
  \begin{equation}
  L_{(f,1)} T_u(x,t) = F(T_u(x,t),u,h(x,u)), \quad h(x,u) = H(T_u(x,t),u),
  \label{eq:meta-observer-transform}
  \end{equation}
  where \(L_{(f,1)} T_u(x,t)\) is the Lie derivative of \(T_u\) along the vector field \((f,1)\)
  \[
  L_{(f,1)} T_u(x,t) = \lim_{\tau \rightarrow 0} \frac{ T_u (X(x,t;t+\tau;u),t+\tau) - T_u(x,t) }{\tau}.
  \]
\item
  \(T_u\) respects the causality condition
  \[
  \forall \tilde{u}: [0,+\infty) \rightarrow \mathbb{R}^m, \forall t \in [0,+\infty), u_{[0,t]} = \tilde{u}_{[0,t]} \Rightarrow T_u(\cdot,t) = T_{\tilde{u}}(\cdot,t).
  \]
\end{enumerate}

Then, for any \(u \in \mathcal{U}\), there exists a function \(\mathcal{T}_u: \mathbb{R}^p \times [0,+\infty) \rightarrow \mathcal{X}\) (satisfying the causality condition) such that for any \(t \geq \bar{t}\), \(\xi \mapsto \mathcal{T}_u (\xi, t)\) is uniformly continuous on \(\mathbb{R}^p\) and satisfies
\[
\mathcal{T}_u \left( T_u(x,t),t \right) = x, \forall x \in \mathcal{X}.
\]
Moreover, denoting \(\mathcal{T}\) the family of functions \(\mathcal{T}_u\) for \(u \in \mathcal{U}\), the couple \((\mathcal{F}, \mathcal{T})\) is an observer for the system \eqref{eq:output-feedback-system} initialized in \(\mathcal{X}_0\).
\end{theorem}

\begin{proof}
See Theorem 1.1 in \citep{bernard19book-observer}.
\end{proof}

A simpler version of Theorem \ref{thm:observerdesignmeta} where the coordinate transformation \(T_u\) is stationary and fixed for all \(u\) is stated below as a corollary.

\begin{corollary}[Meta Observer with Fixed Transformation]
\protect\hypertarget{cor:observerdesignmetafixedT}{}\label{cor:observerdesignmetafixedT}Let \(F: \mathbb{R}^p \times \mathbb{R}^m \times \mathbb{R}^d \rightarrow \mathbb{R}^p\), \(H: \mathbb{R}^p \times \mathbb{R}^m \rightarrow \mathbb{R}^d\) and \(\mathcal{F}: \mathbb{R}^p \times \mathbb{R}^m \times \mathbb{R}^d \rightarrow \mathbb{R}^p\) be continuous functions such that \eqref{eq:meta-observer-zeta-hat} is an observer for \eqref{eq:meta-observer-zeta}.

Suppose there exists a continuous coordinate transformation \(T: \mathbb{R}^p \rightarrow \mathbb{R}^n\) and a compact subset \(\Omega\) of \(\mathbb{R}^n\) such that

\begin{enumerate}
\def\labelenumi{\arabic{enumi}.}
\item
  For any \(x_0 \in \mathcal{X}_0\) such that \(\sigma^+_{\mathbb{X}}(x_0;u) = + \infty\), \(X(x_0;\cdot;u)\) remains in \(\Omega\)
\item
  \(x \mapsto T(x)\) is injective on \(\Omega\)
\item
  \(T\) transforms the system \eqref{eq:output-feedback-system} into system \eqref{eq:meta-observer-zeta}
  \[
  L_f T(x) = F(T(x),u,h(x,u)), \quad h(x,u) = H(T(x),u),
  \]
  where \(L_f T(x)\) is the Lie derivative of \(T(x)\) along \(f\)
  \[
  L_f T(x) = \lim_{\tau \rightarrow 0} \frac{ T(X(x,t;t+\tau;u))  - T(x)}{\tau}.
  \]
\end{enumerate}

Then, there exists a uniformly continuous function \(\mathcal{T}:\mathbb{R}^p \rightarrow \mathbb{R}^{n}\) such that
\[
\mathcal{T}(T(x)) = x, \quad \forall x \in \Omega,
\]
and \((\mathcal{F},\mathcal{T})\) is an observer for system \eqref{eq:output-feedback-system} initialized in \(\mathcal{X}_0\).
\end{corollary}

Theorem \ref{thm:observerdesignmeta} and Corollary \ref{cor:observerdesignmetafixedT} suggest the following general observer design strategy:

\begin{enumerate}
\def\labelenumi{\arabic{enumi}.}
\item
  Find an injective coordinate transformation \(T_u\) (that may be time-varying and also dependent on \(u\)) that transforms the original system \eqref{eq:output-feedback-system} with coordinate \(x\) into a new system \eqref{eq:meta-observer-zeta} with coordinate \(\xi\)
\item
  Design an observer \eqref{eq:meta-observer-zeta-hat}, \(\hat{\xi}\), for the new system
\item
  Compute a left inverse, \(\mathcal{T}_u\), of the transformation \(T_u\) to recover a state estimation \(\hat{x}\) of the original system.
\end{enumerate}

The transformed systems \eqref{eq:meta-observer-zeta} are typically referred to as \emph{normal forms}, or in my opinion, \emph{templates}.

Of course, the general design strategy is rather conceptual, and in order for it to be practical, we have to answer three questions.

\begin{itemize}
\item
  What templates do we have, what are their associated observers, and what are the conditions for the observers to be asymptotically converging?
\item
  What kinds of (nonlinear) systems can be transformed into the templates, and how to perform the transformation?
\item
  How to invert the coordinate transformation? Is it analytical or does it require numerical approximation?
\end{itemize}

In the following sections, we will study several representative normal forms and answer the above questions.

\hypertarget{luenberger-template}{%
\subsection{Luenberger Template}\label{luenberger-template}}

Consider an instance of the normal form \eqref{eq:meta-observer-zeta} as follows:
\begin{equation}
\dot{\xi} = A \xi + B(u,y), \quad y = C \xi,
\label{eq:Luenberger-linear-template}
\end{equation}
where \(A,C\) are constant matrices, and \(B(u,y)\) can depend nonlinearly on \(u\) and \(y\).

For this template, we have the well-known Luenberger observer.

\begin{theorem}[Luenberger Observer]
\protect\hypertarget{thm:LuenbergerLinear}{}\label{thm:LuenbergerLinear}If the pair \((A,C)\) is detectable (see Theorem \ref{thm:ltidetectable}), then there exists a matrix \(K\) such that \(A-KC\) is Hurwitz and the system
\begin{equation}
\dot{\hat{\xi}} = A \hat{\xi} + B(u,y) + K(y - C \hat{\xi})
\label{eq:Luenberger-Linear}
\end{equation}
is an observer for \eqref{eq:Luenberger-linear-template}.
\end{theorem}

\begin{proof}
{TBD.}
\end{proof}

The Luenberger observer is an elegant result in observer design (and even in control theory) that has far-reaching impact. In my opinion, the essence of observer design is twofold: (i) to simulate the dynamics when the state estimation is correct, and (ii) to correct the state estimation from observation when it is off. These two pieces of ideas are evident in \eqref{eq:Luenberger-Linear}: the observer is a copy of the original dynamics (\(A \hat{\xi} + B(u,y)\)) plus a feedback correction from the difference between the ``imagined'' observation \(C\hat{\xi}\) and the true observation \(y\).

You may think the Luenberger template is restricting because it requires the system to be linear (up to the only nonlinearly in \(B(u,y)\)). However, it turns out the Luenberger template is already quite useful, as I will show in the following pendulum example.

\begin{example}[Luenberger Observer for A Simple Pendulum]
\protect\hypertarget{exm:pendulumLuenberger}{}\label{exm:pendulumLuenberger}Consider a simple pendulum dynamics model
\begin{equation}
x = \begin{bmatrix}
\theta \\ \dot{\theta}
\end{bmatrix}, \quad 
\dot{x} = \begin{bmatrix} 
\dot{\theta} \\
- \frac{1}{ml^2} (b \dot{\theta} + mgl \sin \theta) 
\end{bmatrix} + 
\begin{bmatrix}
0 \\
\frac{1}{ml^2} 
\end{bmatrix} u, \quad y = \theta,
\label{eq:kbobserver-pendulum}
\end{equation}
where \(\theta\) the angular position of the pendulum from the vertical line, \(m > 0\) the mass of the pendulum, \(l > 0\) the length, \(g\) the gravitational constant, \(b > 0\) the dampling coefficient, and \(u\) the control input (torque).

We assume we can only observe the angular position of the pendulum in \eqref{eq:kbobserver-pendulum}, e.g., using a camera, but not the angular velocity. Our goal is to construct an observer that can provide a full state estimation.

We first note that the pendulum dynamics \eqref{eq:kbobserver-pendulum} can actually be written in the (linear) Luenberger template \eqref{eq:Luenberger-linear-template} as\footnote{I have to say I was a bit surprised when I arrived at this formulation.}
\begin{equation}
\begin{split}
\dot{x} & = \underbrace{\begin{bmatrix}
0 & 1 \\
0 & - \frac{b}{ml^2}
\end{bmatrix}}_{=:A} x + 
\underbrace{\begin{bmatrix}
0 \\
\frac{u - mgl \sin \theta }{ml^2}
\end{bmatrix}}_{=:B(u,y)} \\
y & = \underbrace{\begin{bmatrix}
1 & 0
\end{bmatrix}}_{=:C} x 
\end{split}.
\label{eq:pendulum-state-affine}
\end{equation}

In order for us to use the Luenberger observer, we need to check if the pair \((A,C)\) is detectable. We can easily find the eigenvalues and eigenvectors of \(A\):
\[
A \begin{bmatrix}
1 \\ 0
\end{bmatrix} = 0,\quad 
A \begin{bmatrix}
- \frac{ml^2}{b} \\ 1 
\end{bmatrix} = \begin{bmatrix}
1 \\ - \frac{b}{ml^2}
\end{bmatrix}
= - \frac{b}{ml^2} \begin{bmatrix}
- \frac{ml^2}{b} \\ 1 \end{bmatrix}.
\]
The first eigenvalue has real part equal to \(0\). However,
\[
C \begin{bmatrix}
1 \\ 0
\end{bmatrix} = 1 \neq 0.
\]
According to Theorem \ref{thm:ltidetectable}, we conclude \((A,C)\) is detectable.

Now we need to find \(K\). An easy choice of \(K\) is
\[
K = \begin{bmatrix} k \\ 0 \end{bmatrix}, \quad A - KC = 
\begin{bmatrix}
- k & 1 \\ 0 & - \frac{b}{ml^2}
\end{bmatrix}.
\]
With \(k > 0\), we know \(A- KC\) is guaranteed to be Hurwitz (the two eigenvalues of \(A-KC\) are \(-k\) and \(-b/ml^2\)), and we have obtained an observer!

{TBD: simulate the observer.}
\end{example}

\hypertarget{state-affine-template}{%
\subsection{State-affine Template}\label{state-affine-template}}

Consider an instance of the normal form \eqref{eq:meta-observer-zeta} where the dynamics is linear in \(\xi\), but the coefficients are time-varying and dependent on the input and output
\begin{equation}
\dot{\xi} = A(u,y) \xi + B(u,y), \quad y = C(u) \xi.
\label{eq:state-affine-template}
\end{equation}
The difference between the state-affine template \eqref{eq:state-affine-template} and the Luenberger template \eqref{eq:Luenberger-linear-template} is that the linear matrices \(A,C\) are allowed to depend nonlinearly on the input \((u,y)\).

Kalman and Bucy originally proposed an observer for linear time-varying systems \citep{kalman61-new}. The result is later extened by \citep{besanccon96ejc-observer} and \citep{hammouri90cdc-observer} to deal with coefficient matrices dependent on the control. The following theorem is a direct extension of the result from \citep{besanccon96ejc-observer} and \citep{hammouri90cdc-observer} by considering \((u,y)\) as an augmented control input.

Before presenting the theorem, we need to introduce the following terminology.

\begin{definition}[Linear Time-Varying System]
\protect\hypertarget{def:lineartimevarying}{}\label{def:lineartimevarying}

For a linear time-varying system of the form
\begin{equation}
\dot{\chi} = A(\nu) \chi, \quad y = C(\nu) \chi,
\label{eq:linear-time-varying}
\end{equation}
with input \(\nu\) and output \(y\), we define

\begin{itemize}
\item
  the \emph{transition matrix} \(\Psi_\nu\) as the unique solution to
  \[
  \Psi_\nu (t,t) = I, \quad \frac{\partial \Psi_\nu}{\partial \tau}(\tau,t) = A(\nu(\tau)) \Psi_\nu (\tau, t).
  \]
  Note that the transition matrix is used to express the solution to \eqref{eq:linear-time-varying} because it satisfies
  \[
  \chi(\chi_0,t_0;t;\nu) = \Psi_\nu (t,t_0) \chi_0.
  \]
\item
  the \emph{observability grammian} as
  \[
  \Gamma_\nu (t_0,t_1) = \int_{t_0}^{t_1} \Psi_\nu (\tau,t_0)^T C(\nu(\tau))^T C(\nu(\tau)) \Psi_\nu (\tau,t_0) d\tau.
  \]
\item
  the \emph{backward observability grammian} as
  \[
  \Gamma_\nu^b (t_0,t_1) = \int_{t_0}^{t_1} \Psi_\nu (\tau,t_1)^T C(\nu(\tau))^T C(\nu(\tau)) \Psi_\nu (\tau,t_1) d\tau.
  \]
\end{itemize}

\end{definition}

We now introduce the Kalman-Bucy Observer for the state-affine template \eqref{eq:state-affine-template}.

\begin{theorem}[Kalman-Bucy Observer]
\protect\hypertarget{thm:kalmanbucystateaffine}{}\label{thm:kalmanbucystateaffine}Let \(y_{\xi_0,u}(t) = C(u(t)) \Xi (\xi_0;t;u)\) be the output of system \eqref{eq:state-affine-template} at time \(t\) with initialization \(\xi_0\) and control \(u\). Suppose the control \(u\) satisfies

\begin{itemize}
\item
  For any \(\xi_0\), \(t \mapsto A(u(t),y_{\xi_0,u}(t))\) is bounded by \(A_{\max}\)
\item
  For any \(\xi_0\), the augmented input \(\nu = (u,y_{\xi_0,u})\) is \emph{regularly persistent} for the dynamics
  \begin{equation}
  \dot{\chi} = A(\nu) \chi , \quad y = C(\nu) \chi 
  \label{eq:kbobserver-auxilarydynamics}
  \end{equation}
  uniformly with respect to \(\xi_0\). That is, there exist strictly positive numbers \(t_0,\bar{t}\), and \(\alpha\) such that for any \(\xi_0\) and any time \(t \geq t_0 \geq \bar{t}\),
  \[
  \Gamma_v^b (t-\bar{t}, t) \succeq \alpha I,
  \]
  where \(\Gamma_v^b\) is the \emph{backward observability grammian} associated with system \eqref{eq:kbobserver-auxilarydynamics}.
\end{itemize}

Then, given any positive definite matrix \(P_0\), there exist \(\alpha_1,\alpha_2 > 0\) such that for any \(\lambda \geq 2 A_{\max}\) and any \(\xi_0 \in \mathbb{R}^p\), the matrix differential equation
\begin{equation}
\dot{P} = -\lambda P - A(u,y)^T P - P A(u,y) + C(u)^T C(u)
\label{eq:kbobserver-matrixdifferential}
\end{equation}
initialized at \(P(0) = P_0\) admits a unique solution satisfying \(P(t)=P(t)^T\) and
\[
\alpha_2 I \succeq P(t) \succeq \alpha_1 I.
\]
Moreover, the system
\begin{equation}
\dot{\hat{\xi}} = A(u,y) \hat{\xi} + B(u,y) + K (y - C(u)\hat{\xi})
\label{eq:kbobserver-observer}
\end{equation}
with a time-varying gain matrix
\begin{equation}
K = P^{-1} C(u)^T 
\label{eq:kbobserver-gain}
\end{equation}
is an observer for the state-affine system \eqref{eq:state-affine-template}.
\end{theorem}

Let us work out an example of the Kalman-Bucy Observer for nonlinear systems.

\begin{example}[Kalman-Bucy Observer for A Simple Pendulum]
\protect\hypertarget{exm:pendulumkbobserver}{}\label{exm:pendulumkbobserver}Let us reconsider the pendulum dynamics \eqref{eq:kbobserver-pendulum} but this time try to design a Kalman-Bucy observer.

We first write the pendulum dynamics in a new coordinate system so that it is in the state-affine normal form \eqref{eq:state-affine-template}. We choose \(\xi = [\mathfrak{s},\mathfrak{c},\dot{\theta}]^T\) with \(\mathfrak{s} = \sin \theta\) and \(\mathfrak{c} = \cos \theta\). Clearly, we will be able to observe \(y = [\mathfrak{s},\mathfrak{c}]^T\) in this new coordinate. The state-affine normal form of the pendulum dynamics reads
\begin{equation}
\begin{split}
\dot{\xi} = \begin{bmatrix}
\mathfrak{c} \dot{\theta} \\
- \mathfrak{s} \dot{\theta} \\
- \frac{1}{ml^2} (b \dot{\theta} + mgl \mathfrak{s} ) +  \frac{1}{ml^2} u
\end{bmatrix} & = 
\underbrace{\begin{bmatrix}
0 & 0 & \mathfrak{c} \\
0 & 0 & -\mathfrak{s} \\
0 & 0 & -\frac{b}{ml^2}
\end{bmatrix}}_{=:A(u,y)} \xi + 
\underbrace{\begin{bmatrix}
0 \\ 0 \\ \frac{u - mgl \mathfrak{s}}{ml^2}
\end{bmatrix}}_{=:B(u,y)} \\
y & = \underbrace{\begin{bmatrix}
1 & 0 & 0 \\
0 & 1 & 0
\end{bmatrix}}_{=:C(u)} \xi
\end{split}.
\label{eq:pendulum-state-affine}
\end{equation}
Note that \(C(u)\) is in fact time-invariant, and \(B(u,y)\) only depends on \(u\); but we adopt the same notation as the general state-affine template \eqref{eq:state-affine-template}.

In order to use the Kalman-Bucy observer in Theorem \ref{thm:kalmanbucystateaffine}, we need to verify the boundedness of \(A(u,y)\), and the regular persistence of \eqref{eq:kbobserver-auxilarydynamics}.

\textbf{Boundedness of \(A(u,y)\)}. We can easily show the boundedness of \(A(u,y)\) by writing
\[
\Vert A(u,y) \xi \Vert = \Vert \xi_3 (\mathfrak{c} - \mathfrak{s} - b/ml^2) \Vert  \leq |\xi_3| \sqrt{3} \sqrt{\mathfrak{c}^2 + \mathfrak{s}^2 + b^2 / m^2 l^4} \leq \Vert \xi \Vert \sqrt{3 + 3b^2 / m^2 l^4}. 
\]
Therefore, we can take \(A_{\max} = \sqrt{3 + 3b^2 / m^2 l^4}\).
{Does the \(A_{\max}\) in Theorem \ref{thm:kalmanbucystateaffine} refer to the bound in operator norm?}

\textbf{Regular persistence}. We write the backward observability grammian of system \eqref{eq:kbobserver-auxilarydynamics}
\[
\Gamma_\nu^b(t - \bar{t},t) = \int_{t - \bar{t}}^t \Psi_\nu(\tau,t)^T C^T C \Psi_\nu (\tau, t) d \tau = \int_{t - \bar{t}}^t \Psi_\nu(\tau,t)^T \begin{bmatrix} 
1 & 0 & 0 \\
0 & 1 & 0 \\
0 & 0 & 0
\end{bmatrix}
\Psi_\nu (\tau, t) d \tau
\]
{Does the transition matrix \(\Psi_\nu (\tau, t)\) have an analytical form?}
\end{example}

\hypertarget{triangular-template}{%
\subsection{Triangular Template}\label{triangular-template}}

\hypertarget{adaptivecontrol}{%
\chapter{Adaptive Control}\label{adaptivecontrol}}

\hypertarget{model-reference-adaptive-control}{%
\section{Model-Reference Adaptive Control}\label{model-reference-adaptive-control}}

Basic flow for designing an adaptive controller

\begin{enumerate}
\def\labelenumi{\arabic{enumi}.}
\item
  Design a control law with variable parameters
\item
  Design an adaptation law for adjusting the control parameters
\item
  Analyze the convergence of the closed-loop system
\end{enumerate}

The control law design at the first step typically requires the designer to know what a good controller is if the true parameters were actually known, e.g., from feedback linearization (Appendix \ref{feedbacklinearization}), sliding control (Appendix \ref{slidingcontrol}) etc.

The design of the adaptation law typically comes from analyzing the dynamics of the tracking error, which as we will see often appears in the form of Lemma \ref{lem:adaptivecontrolbasic}.

The convergence of the closed-loop system is usually analyzed with the help of a Lyapunov-like function introduced in Chapter \ref{stability}.

\begin{lemma}[Basic Lemma]
\protect\hypertarget{lem:adaptivecontrolbasic}{}\label{lem:adaptivecontrolbasic}Let two signals \(e(t)\) and \(\phi(t)\) be related by
\begin{equation}
e(t) = H(p)[k \phi(t)^T v(t)]
\label{eq:acbasiclemmaephi}
\end{equation}
where \(e(t)\) a scalar output signal, \(H(p)\) a strictly positive real (SPR) transfer function, \(k\) an unknown real number with known sign, \(\phi(t) \in \mathbb{R}^m\) a control signal, and \(v(t) \in \mathbb{R}^m\) a measurable input signal.

If the control signal \(\phi(t)\) satisfies
\begin{equation}
\dot{\phi}(t) = - \mathrm{sgn}(k) \gamma e(t) v(t)
\label{eq:acbasiclemmaphilaw}
\end{equation}
with \(\gamma > 0\) a positive constant, then \(e(t)\) and \(\phi(t)\) are globally bounded. Moreover, if \(v(t)\) is bounded, then
\[
\lim_{t \rightarrow \infty} e(t) = 0.
\]
\end{lemma}

\begin{proof}
Let the state-space representation of \eqref{eq:acbasiclemmaephi} be
\begin{equation}
\dot{x} = A x + b [k \phi^T v], \quad e = c^T x.
\label{eq:acbasiclemmastatespace}
\end{equation}
Since \(H(p)\) is SPR, it follows from the Kalman-Yakubovich Lemma \ref{lem:KalmanYakubovich} that there exist \(P,Q \succ 0\) such that
\[
A^T P + P A = -Q, \quad Pb = c.
\]
Let
\[
V(x,\phi) = x^T P x + \frac{|k|}{\gamma} \phi^T \phi,
\]
clearly \(V\) is positive definite (i.e., \(V(0,0)=0\), and \(V(x,\phi) > 0\) for all \(x \neq 0, \phi \neq 0\)). The time derivative of \(V\) along the trajectory defined by \eqref{eq:acbasiclemmastatespace} with \(\phi\) chosen as in \eqref{eq:acbasiclemmaphilaw} is
\begin{align}
\dot{V} & = \frac{\partial V}{\partial x} \dot{x} + \frac{\partial V}{\partial \phi} \dot{\phi} \\
&= x^T (PA + A^T P) x + 2 x^T P b (k \phi^T v) + \frac{2|k|}{\gamma} \phi^T (- \mathrm{sgn}(k) \gamma e v) \\
& = - x^T Q x + 2 (x^T c)(k\phi^T v) - 2 \phi^T (e v) \\
& = - x^T Q x \leq 0.
\end{align}
As a result, we know \(x\) and \(\phi\) must be bounded (\(V(x(t),\phi(t)) \leq V(x(0),\phi(0))\) is bounded). Since \(e = c^T x\), we know \(e\) must be bounded as well.

If the input signal \(v\) is also bounded, then \(\dot{x}\) is bounded as seen from \eqref{eq:acbasiclemmastatespace}. Because \(\ddot{V} = -2x^T Q \dot{x}\) is now bounded, we know \(\dot{V}\) is uniformly continuous. Therefore, by Barbalat's stability certificate (Theorem \ref{thm:BarbalatStability}), we know \(\dot{V}\) tends to zero as \(t\) tends to infinity, which implies \(\lim_{t \rightarrow \infty} x(t) = 0\) and hence \(\lim_{t \rightarrow \infty} e(t) = 0\).
\end{proof}

\hypertarget{first-order-systems}{%
\subsection{First-Order Systems}\label{first-order-systems}}

Consider the first-order single-input single-output (SISO) system
\begin{equation}
\dot{x} = - a x + b u
\label{eq:ac-first-linear}
\end{equation}
where \(a\) and \(b\) are unknown groundtruth parameters. However, we do assume that the sign of \(b\) is known. {What if the sign of \(b\) is unknown too?}

Let \(r(t)\) be a reference trajectory, e.g., a step function or a sinusoidal function, and \(x_d(t)\) be a desired system trajectory that tracks the reference
\begin{equation}
\dot{x}_d = - a_d x_d + b_d r(t),
\label{eq:ac-first-linear-desired}
\end{equation}
where \(a_d,b_d > 0\) are user-defined constants. Note that the transfer function from \(r\) to \(x_d\) is
\[
x_d = \frac{b_d}{p + a_d} r
\]
and the system is stable. {Review basics of transfer function.}

The goal of adaptive control is to design a control law and an adaptation law such that the tracking error of the system \(x(t) - x_d(t)\) converges to zero.

\textbf{Control law}. We design the control law as
\begin{equation}
u = \hat{a}_r(t) r + \hat{a}_x(t) x 
\label{eq:ac-first-linear-control}
\end{equation}
where \(\hat{a}_r(t)\) and \(\hat{a}_x(t)\) are time-varying feedback gains that we wish to adapt. The closed-loop dynamics of system \eqref{eq:ac-first-linear} with the controller \eqref{eq:ac-first-linear-control} is
\[
\dot{x} = - a x + b (\hat{a}_r r + \hat{a}_x x) = - (a - b \hat{a}_x) x + b \hat{a}_r r. 
\]
With the equation above,
the reason for choosing the control law \eqref{eq:ac-first-linear-control} is clear: if the system parameters \((a,b)\) were known, then choosing
\begin{equation}
a_r^\star = \frac{b_d}{b}, \quad a_x^\star = \frac{a - a_d}{b}
\label{eq:ac-first-linear-optimal-gain}
\end{equation}
leads to the closed-loop dynamics \(\dot{x} = - a_d x + b_d r\) that is exactly what we want in \eqref{eq:ac-first-linear-desired}.

However, in adaptive control, since the true parameters \((a,b)\) are not revealed to the control designer, an adaptation law is needed to dynamically adjust the gains \(\hat{a}_r\) and \(\hat{a}_x\) based on the tracking error \(x(t) - x_d(t)\).

\textbf{Adaptation law}. Let \(e(t) = x(t) - x_d(t)\) be the tracking error, and we develop its time derivative
\begin{align}
\dot{e} &= \dot{x} - \dot{x}_d \\
        &= - a_d (x - x_d) + (a_d - a + b\hat{a}_x)x + (b \hat{a}_r - b_d) r \\
        & = - a_d e + b\underbrace{(\hat{a}_x - \hat{a}_x^\star)}_{=:\tilde{a}_x} x + b \underbrace{(\hat{a}_r - \hat{a}_r^\star )}_{=:\tilde{a}_r} r \\
        & = - a_d e + b (\tilde{a}_x x + \tilde{a}_r r) \label{eq:ac-first-linear-error-dynamics}
\end{align}
where \(\tilde{a}_x\) and \(\tilde{a}_r\) are the gain errors w.r.t. the optimal gains in \eqref{eq:ac-first-linear-optimal-gain} if the true parameters were known. The error dynamics \eqref{eq:ac-first-linear-error-dynamics} is equivalent to the following transfer function
\begin{equation}
e = \frac{1}{p + a_d} b(\tilde{a}_x x + \tilde{a}_r r) = \frac{1}{p + a_d} \left(b 
\begin{bmatrix} \tilde{a}_x \\ \tilde{a}_r \end{bmatrix}^T 
\begin{bmatrix} x \\ r \end{bmatrix}
\right),
\label{eq:ac-first-linear-error-dynamics-transfer}
\end{equation}
which is in the form of \eqref{eq:acbasiclemmaephi}. Therefore, we choose the adaptation law
\begin{equation}
\begin{bmatrix} \dot{\tilde{a}}_x \\ \dot{\tilde{a}}_r \end{bmatrix} = - \mathrm{sgn}(b) \gamma e \begin{bmatrix} x \\ r \end{bmatrix}.
\label{eq:ac-first-linear-adaptation-law}
\end{equation}

\textbf{Tracking convergence}. With the control law \eqref{eq:ac-first-linear-control} and the adaptation law \eqref{eq:ac-first-linear-adaptation-law}, we can prove that the tracking error converges to zero, using Lemma \ref{lem:adaptivecontrolbasic}. With \(\tilde{a}=[\tilde{a}_x, \tilde{a}_r]^T\), let
\begin{equation}
V(e,\tilde{a}) = e^2 + \frac{|b|}{\gamma} \tilde{a}^T \tilde{a}
\label{eq:ac-first-linear-lyapunov}
\end{equation}
be a positive definite Lyapunov function candidate with time derivative
\[
\dot{V} = - 2a_d e^2 \leq 0.
\]
Clearly, \(e\) and \(\tilde{a}\) are both bounded. Assuming the reference trajectory \(r\) is bounded, we know \(x_d\) is bounded (due to \eqref{eq:ac-first-linear-desired}) and hence \(x\) is bounded (due to \(e = x - x_d\) being bounded). Consequently, from the error dynamics \eqref{eq:ac-first-linear-error-dynamics} we know \(\dot{e}\) is bounded, which implies \(\ddot{V} = -4a_d e \dot{e}\) is bounded and \(\dot{V}\) is uniformly continuous. By Barbalat's stability certificate \ref{thm:BarbalatStability}, we conlude \(e(t) \rightarrow 0\) as \(t \rightarrow \infty\).

It is always better to combine mathematical analysis with intuitive understanding. Can you explain intuitively why the adaptation law \eqref{eq:ac-first-linear-adaptation-law} makes sense? (Hint: think about how the control should react to a negative/positive tracking error.)

\textbf{Parameter convergence}. We have shown the control law \eqref{eq:ac-first-linear-control} and the adaptation law \eqref{eq:ac-first-linear-adaptation-law} guarantee to track the reference trajectory. However, is it guaranteed that the gains of the controller \eqref{eq:ac-first-linear-control} also converge to the optimal gains in \eqref{eq:ac-first-linear-optimal-gain}?

We will now show that the answer is indefinite and it depends on the reference trajectory \(r(t)\). Because the tracking error \(e\) converges to zero, and \(e\) is the output of a stable filter \eqref{eq:ac-first-linear-error-dynamics-transfer}, we know the input \(b(\tilde{a}_x x + \tilde{a}_r r)\) must also converge to zero. On the other hand, the adaptation law \eqref{eq:ac-first-linear-adaptation-law} shows that both \(\dot{\tilde{a}}_x\) and \(\dot{\tilde{a}}_r\) converge to zero (due to \(e\) converging to zero and \(x\), \(r\) being bounded). As a result, we know \(\tilde{a} = [\tilde{a}_x,\tilde{a}_r]^T\) converges to a constant that satisfies
\begin{equation}
v^T \tilde{a} = 0, \quad v = \begin{bmatrix} x \\ r \end{bmatrix},
\label{eq:ac-first-linear-parameter-equation}
\end{equation}
which is a single linear equation of \(\tilde{a}\) with time-varying coeffients.

\begin{itemize}
\item
  \textbf{Constant reference: no guaranteed convergence}. Suppose \(r(t) \equiv r_0 \neq 0\) for all \(t\). From \eqref{eq:ac-first-linear-desired} we know \(x = x_d = \alpha r_0\) when \(t \rightarrow \infty\), where \(\alpha\) is the constant DC gain of the stable filter. Therefore, the linear equation \eqref{eq:ac-first-linear-parameter-equation} reduces to
  \[
  \alpha \tilde{a}_x + \tilde{a}_r  = 0. 
  \]
  This implies that \(\tilde{a}\) does not necessarily converge to zero. In fact, it converges to a straight line in the parameter space.
\item
  \textbf{Persistent excitation: guaranteed convergence}. However, when the signal \(v\) satisfies the so-called \emph{persistent excitation} condition, which states that for any \(t\), there exists \(T, \beta > 0\) such that
  \begin{equation}
  \int_{t}^{t+T} v v^T d\tau \geq \beta I,
  \label{eq:ac-first-linear-persistent-excitation}
  \end{equation}
  then \(\tilde{a}\) is guaranteed to converge to zero. To see this, we multiply \eqref{eq:ac-first-linear-parameter-equation} by \(v\) and integrate it from \(t\) to \(t+T\), which gives rise to
  \[
  \left( \int_{t}^{t+T} vv^T d\tau \right) \tilde{a} = 0.
  \]
  By the persistent excitation condition \eqref{eq:ac-first-linear-persistent-excitation}, we infer that \(\tilde{a} = 0\) is the only solution.
\end{itemize}

It remains to understand under what conditions of the reference trajectory \(r(t)\) can we guarantee the persistent excitation of \(v\). We leave it as an exercise for the reader to show, if \(r(t)\) contains at least one sinusoidal component, then the persistent excitation condition of \(v\) is guaranteed.

\begin{exercise}[Extension to Nonlinear Systems]
\protect\hypertarget{exr:adaptivecontrolfirstordernonlinearsystem}{}\label{exr:adaptivecontrolfirstordernonlinearsystem}Design a control law and an adaptation law for the following system
\[
\dot{x} = - a x - c f(x) + b u
\]
with unknown true parameters \((a,b,c)\) (assume the sign of \(b\) is known) and known nonlinearity \(f(x)\) to track a reference trajectory \(r(t)\). Analyze the convergence of tracking error and parameter estimation error.
\end{exercise}

\hypertarget{high-order-systems}{%
\subsection{High-Order Systems}\label{high-order-systems}}

Consider an \(n\)-th order nonlinear system
\begin{equation}
q^{(n)} + \sum_{i=1}^n \alpha_i f_i(x,t) = bu
\label{eq:ac-singleinput-nonlinear}
\end{equation}
where \(x=[q,\dot{q},\ddot{q},\dots,q^{(n-1)}]^T\) is the state of the system, \(f_i\)'s are known nonlinearities, \((\alpha_1,\dots,\alpha_n,b)\) are unknown parameters of the system (with \(\mathrm{sgn}(b)\) known).

The goal of adaptive control is to control the system \eqref{eq:ac-singleinput-nonlinear} trajectory to follow a desired trajectory \(q_d(t)\) despite no knowing the true parameters.

To facilitate the derivation of the adaptive controller, let us divide both sides of \eqref{eq:ac-singleinput-nonlinear} by \(b\)
\begin{equation}
h q^{(n)} + \sum_{i=1}^n a_i f_i(x,t) = u
\label{eq:ac-singleinput-nonlinear-equivalent}
\end{equation}
where \(h = 1 / b\) and \(a_i = \alpha_i / b\).

\textbf{Control law}. Recall that the choice of the control law is tyically inspired by the control design if the true system parameters were known. We will borrow ideas from sliding control (Appendix \ref{slidingcontrol}).

\begin{itemize}
\item
  \textbf{Known parameters}. Let \(e = q(t) - q_d(t)\) be the tracking error, and define the following combined error
  \[
  s = e^{(n-1)} + \lambda_{n-2} e^{(n-2)} + \dots + \lambda_0 e = \Delta(p) e 
  \]
  where \(\Delta(p) = p^{n-1} + \lambda_{n-2} p^{(n-2)} + \dots + \lambda_0\) is a stable polynomial with user-chosen coeffients \(\lambda_0,\dots,\lambda_{n-2}\). The rationale for defining the combined error \(s\) is that the convergence of \(e\) to zero can be guaranteed by the convergence of \(s\) to zero (when \(\Delta(p)\) is stable).
  Note that \(s\) can be equivalently written as
  \begin{align}
  s & = (q^{(n-1)} - q_d^{(n-1)}) + \lambda_{n-2} e^{(n-2)} + \dots + \lambda_0 e \\
  & = q^{(n-1)} - \underbrace{ \left( q_d^{(n-1)} - \lambda_{n-2} e^{(n-2)} - \dots - \lambda_0 e  \right) }_{q_r^{(n-1)}}.
  \end{align}
  Now consider the control law
  \begin{equation}
  u = h q_r^{(n)} - ks + \sum_{i=1}^n a_i f_i(x,t)
  \label{eq:ac-singleinput-nonlinear-control-law-knownparam}
  \end{equation}
  where
  \[
  q_r^{(n)} = q_d^{(n)} - \lambda_{n-2} e^{(n-1)} - \dots - \lambda_0 \dot{e}
  \]
  and \(k\) is a design constant that has the same sign as \(h\). This choice of control, plugged into the system dynamics \eqref{eq:ac-singleinput-nonlinear-equivalent}, leads to
  \begin{align}
  h q^{(n)} + \sum_{i=1}^n a_i f_i(x,t) = h q_r^{(n)} - ks + \sum_{i=1}^n a_i f_i(x,t) \Longleftrightarrow \\
  h \left( q^{(n)} - q_r^{(n)} \right) + ks = 0 \Longleftrightarrow \\
  h \dot{s} + ks = 0,
  \end{align}
  which guarantees the exponential convergence of \(s\) to zero (note that \(h\) and \(k\) have the same sign), and hence the convergence of \(e\) to zero.
\item
  \textbf{Unknown parameters}. Inspired by the control law with known parameters in \eqref{eq:ac-singleinput-nonlinear-control-law-knownparam}, we design the adapative control law as
  \begin{equation}
  u = \hat{h} q_r^{(n)} - ks + \sum_{i=1}^n \hat{a}_i f_i(x,t),
  \label{eq:ac-singleinput-nonlinear-control-law-unknownparam}
  \end{equation}
  where the time-varying gains \(\hat{h},\hat{a}_1,\dots,\hat{a}_n\) will be adjusted by an adaptation law.
\end{itemize}

\textbf{Adaptation law}. Inserting the adapative control law \eqref{eq:ac-singleinput-nonlinear-control-law-unknownparam} into the system dynamics \eqref{eq:ac-singleinput-nonlinear-equivalent}, we obtain
\begin{align}
h \dot{s} + ks = \tilde{h} q_r^{(n)} + \sum_{i=1}^n \tilde{a}_i f_i (x,t) \Longleftrightarrow \\
s = \frac{1}{p + k/h} \frac{1}{h} \underbrace{ \left( 
    \begin{bmatrix} \tilde{h} \\ \tilde{a}_1 \\ \vdots \\ \tilde{a}_n \end{bmatrix}^T
    \begin{bmatrix} q_r^{(n)} \\ f_1(x,t) \\ \vdots \\ f_n(x,t) \end{bmatrix}
    \right)}_{=:\phi^T v}
\label{eq:ac-singleinput-nonlinear-error-dynamics}
\end{align}
where \(\tilde{h} = \hat{h} - h\) and \(\tilde{a}_i = \hat{a}_i - a_i,i=1,\dots,n\). Again, \eqref{eq:ac-singleinput-nonlinear-error-dynamics} is in the familiar form of \eqref{eq:acbasiclemmaephi}, which naturally leads to the following adaptation law with \(\gamma > 0\) a chosen constant
\begin{equation}
\dot{\phi} = \begin{bmatrix} \dot{\tilde{h}} \\ \dot{\tilde{a}}_1 \\ \vdots \\ \dot{\tilde{a}}_n \end{bmatrix} = - \mathrm{sgn}(h) \gamma s \begin{bmatrix} q_r^{(n)} \\ f_1(x,t) \\ \vdots \\ f_n(x,t) \end{bmatrix}.
\label{eq:ac-singleinput-nonlinear-adaptation-law}
\end{equation}

\textbf{Tracking and parameter convergence}. With the following Lyapunov function
\begin{equation}
V(s,\phi) = |h| s^2 + \frac{1}{\gamma} \phi^T \phi, \quad \dot{V}(s,\phi) -2|k| s^2,
\label{eq:ac-singleinput-nonlinear-lyapunov}
\end{equation}
the global convergence of \(s\) to zero can be easily shown. For parameter convergence, it is easy to see that when \(v\) satisfies the persistent excitation condition, we have that \(\phi\) converges to zero. (However, the relationship between the reference trajectory \(q_d(t)\) and the persistent excitation of \(v\) becomes nontrivial due to the nonlinearities \(f_i\).)

\hypertarget{robotic-manipulator}{%
\subsection{Robotic Manipulator}\label{robotic-manipulator}}

So far our focus has been on systems with a single input (\(u \in \mathbb{R}\)). In the following, we will show that similar techniques can be applied to adapative control of systems with multiple inputs, particularly, trajectory control of a robotic manipulator.

Let \(q \in \mathbb{R}^n\) be the joint angles of a multi-link robotic arm, and \(\dot{q} \in \mathbb{R}^n\) be the joint velocities. The dynamics of a robotic manipulator reads
\begin{equation}
H(q) \ddot{q} + C(q,\dot{q})\dot{q} + g(q) = \tau,
\label{eq:ac-mrac-manipulator-dynamics}
\end{equation}
where \(H(q) \in \mathbb{S}^{n}_{++}\) is the manipulator inertia matrix (that is positive definite), \(C(q,\dot{q})\dot{q}\) is a vector of centripetal and Coriolis torques (with \(C(q,\dot{q}) \in \mathbb{R}^{n \times n}\)), and \(g(q)\) denotes gravitational torques.

\begin{figure}

{\centering \includegraphics[width=0.6\linewidth]{images/drawme} 

}

\caption{Planar two-link manipulator}\label{fig:planar-two-link}
\end{figure}

\begin{example}[Planar Two-link Manipulator]
\protect\hypertarget{exm:planartwolinkmanipulator}{}\label{exm:planartwolinkmanipulator}The dynamics of a planar two-link manipulator in Fig. \ref{fig:planar-two-link} is
\begin{equation}
\begin{bmatrix} H_{11} & H_{12} \\ H_{21} & H_{22} \end{bmatrix}
\begin{bmatrix} \ddot{q}_1 \\ \ddot{q}_2 \end{bmatrix}
+ 
\begin{bmatrix} - h \dot{q}_2 & -h (\dot{q}_1 + \dot{q}_2) \\ h \dot{q}_1 & 0 \end{bmatrix}
\begin{bmatrix} \dot{q}_1 \\ \dot{q}_2 \end{bmatrix}
=
\begin{bmatrix} \tau_1 \\ \tau_2 \end{bmatrix},
\label{eq:ac-mrac-manipulator-dynamics-planartwolink}
\end{equation}
where
\begin{align}
H_{11} & = a_1 + 2 a_3 \cos q_2 + 2 a_4 \sin q_2 \\
H_{12} & = H_{21} = a_2 + a_3 \cos q_2 + a_4 \sin q_2 \\
H_{22} &= a_2 \\
h &= a_3 \sin q_2 - a_4 \cos q_2 
\end{align}
with
\begin{align}
a_1 &= I_1 + m_1 l_{c1}^2 + I_e + m_e l_{ce}^2 + m_e l_1^2 \\
a_2 &= I_e + m_e l_{ce}^2 \\
a_3 &= m_e l_1 l_{ce} \cos \delta_e \\
a_4 &= m_e l_1 l_{ce} \sin \delta_e.
\end{align}
\end{example}

As seen from the above example, the parameters \(a\) (which are nonlinear functions of the physical parameters such as mass and length) enter linearly in \(H\) and \(C\) (\(g(q)\) is ignored because the manipulator is on a horizontal plane).

The goal of the control design is to have the manipulator track a desired trajectory \(q_d(t)\).

\textbf{Known parameters}. When the parameters are known, we follow the sliding control design framework. Let \(\tilde{q} = q(t) - q_d(t)\) be the tracking error, and define the combined error
\[
s = \dot{\tilde{q}} + \Lambda \tilde{q} = \dot{q} - \underbrace{\left( \dot{q}_d - \Lambda \tilde{q} \right)}_{\dot{q}_r}
\]
where \(\Lambda \in \mathbb{S}^n_{++}\) is a user-chosen positive definite matrix (in general we want \(-\Lambda\) to be Hurwitz). In this case, \(s \rightarrow 0\) implies \(\tilde{q} \rightarrow 0\) as \(t \rightarrow \infty\). Choosing the control law (coming from feedback linearization Appendix \ref{feedbacklinearization})
\begin{equation}
\tau = H \ddot{q}_r - K_D s + C \dot{q} + g(q)
\label{eq:ac-mrac-manipulator-controller-knownparam}
\end{equation}
with \(K_D \in \mathbb{S}^n_{++}\) positive definite leads to the closed-loop dynamics
\[
H \dot{s} + K_D s = 0 \Longleftrightarrow \dot{s} = - H^{-1} K_D s.
\]
Because the matrix \(H^{-1} K_D\) is the product of two positive definite matrices (recall \(H\) is positive definite and so is \(H^{-1}\)), it has strictly positive real eigenvalues.\footnote{Consider two positive definite matrices \(A\) and \(B\), let \(B = B^{1/2}B^{1/2}\). The product \(AB\) can be written as \(AB = A B^{1/2}B^{1/2} = B^{-1/2} (B^{1/2} A B^{1/2}) B^{1/2}\). Therefore \(AB\) is \href{https://en.wikipedia.org/wiki/Matrix_similarity}{similar} to \(B^{1/2} A B^{1/2}\) and is positive definite.} Hence, \(- H^{-1} K_D\) is \href{https://en.wikipedia.org/wiki/Hurwitz_matrix}{Hurwitz} and \(s\) is guaranteed to converge to zero.

\textbf{Control law}. A closer look at the controller \eqref{eq:ac-mrac-manipulator-controller-knownparam} allows us to write it in the following form
\begin{align}
\tau &= H \ddot{q}_r + C(s + \dot{q}_r) + g(q) - K_D s  \\
&= H \ddot{q}_r + C \dot{q}_r + g(q) + (C - K_D) s \\
&= Y (q,\dot{q},\dot{q}_r,\ddot{q}_r) a + (C - K_D) s
\end{align}
where \(a \in \mathbb{R}^m\) contains all the parameters and \(Y \in \mathbb{R}^{n \times m}\) is the matrix that collects all the coeffients of \(a\) in \(H \ddot{q}_r + C \dot{q}_r + g(q)\). As a result, we design the adapative control law to be
\begin{equation}
\tau = Y \hat{a} - K_D s,
\label{eq:ac-mrac-manipulator-controller-unknownparam}
\end{equation}
with \(\hat{a}\) the time-varying parameter that we wish to adapt. Note that here we have done something strange: the adapative control law does not exactly follow the controller \eqref{eq:ac-mrac-manipulator-controller-knownparam} in the known-parameter case.\footnote{In fact, one can show that the controller \eqref{eq:ac-mrac-manipulator-controller-unknownparam} with known parameters, i.e., \(\tau = Y a - K_D s\), also guarantees the convergence of \(s\) towards zero, though it is different from the feedback linearization controller \eqref{eq:ac-mrac-manipulator-controller-knownparam}. Try proving the convergence with a Lyapunov candidate \(V = \frac{1}{2}s^T H s\).} We first separated \(s\) from \(\dot{q}\) and wrote \(Ya = H \ddot{q}_r + C \dot{q}_r + g\) instead of \(Ya = H \ddot{q}_r + C \dot{q} + g\); then we dropped the ``\(C\)'' matrix in front of \(s\) in the adapative control law. The reason for doing this will soon become clear when we analyze the tracking convergence.

\textbf{Adaptation law and tracking convergence}.
Recall that the key of adapative control is to design a control law and an adaptation law such that global converge of the tracking error \(s\) can be guaranteed by a Lyapunov function. Looking at the previous Lyapunov functions in \eqref{eq:ac-first-linear-lyapunov} and \eqref{eq:ac-singleinput-nonlinear-lyapunov}, we see that they both contain a positive definite term in the tracking error \(s\) (or \(e\) if in first-order systems) and another positive definite term in the parameter error \(\tilde{a}\). This hints us that we may try a Lyapunov candidate function of the following form
\begin{equation}
V = \frac{1}{2} \left( s^T H s + \tilde{a} \Gamma^{-1} \tilde{a} \right),
\label{eq:ac-mrac-manipulator-controller-unknownparam-lyapunov}
\end{equation}
where \(\Gamma \in \mathbb{S}^m_{++}\) is a constant positive definite matrix, and \(\tilde{a} = \hat{a} - a\) is the parameter error.

The next step would be to derive the time derivative of \(V\), which, as we can expect, will contain a term that involves \(\dot{H}\) and complicates our analysis. Fortunately, the following lemma will help us.

\begin{lemma}
\protect\hypertarget{lem:skewsymmetricmanipulator}{}\label{lem:skewsymmetricmanipulator}For the manipulator dynamics \eqref{eq:ac-mrac-manipulator-dynamics}, there exists a way to define \(C\) such that \(\dot{H} - 2C\) is skew-symmetric.
\end{lemma}

\begin{proof}
See Section 9.1, page 399-402 in \citep{slotine91book-applied}. You should also check if this is true for the planar two-link manipulator dynamics in Example \ref{exm:planartwolinkmanipulator}.
\end{proof}

With Lemma \ref{lem:skewsymmetricmanipulator}, the time derivative of \(V\) in \eqref{eq:ac-mrac-manipulator-controller-unknownparam-lyapunov} reads
\begin{align}
\dot{V} & = s^T H \dot{s} + \frac{1}{2} s^T \dot{H} s + \tilde{a}^T \Gamma^{-1} \dot{\tilde{a}} \\
&= s^T (H \ddot{q} - H \ddot{q}_r) + \frac{1}{2} s^T \dot{H} s + \tilde{a}^T \Gamma^{-1} \dot{\tilde{a}} \\
&= 
s^T (\tau - C \dot{q} - g - H \ddot{q}_r ) + \frac{1}{2} s^T \dot{H} s + \tilde{a}^T \Gamma^{-1} \dot{\tilde{a}} \label{eq:ac-mrac-manipulator-controller-unknownparam-lyapunov-derivative-1} \\
&= s^T (\tau - H \ddot{q}_r - C (s + \dot{q}_r) - g) + \frac{1}{2} s^T \dot{H} s + \tilde{a}^T \Gamma^{-1} \dot{\tilde{a}} \\
&= s^T (\tau - H \ddot{q}_r - C \dot{q}_r - g) + \frac{1}{2} s^T (\dot{H}- 2C)s + \tilde{a}^T \Gamma^{-1} \dot{\tilde{a}} \label{eq:ac-mrac-manipulator-controller-unknownparam-lyapunov-derivative-2}\\
&= s^T (\tau - H \ddot{q}_r - C \dot{q}_r - g) + \tilde{a}^T \Gamma^{-1} \dot{\tilde{a}} \\
&=s^T(Y\hat{a} - K_D s - Ya) + \tilde{a}^T \Gamma^{-1} \dot{\tilde{a}} \label{eq:ac-mrac-manipulator-controller-unknownparam-lyapunov-derivative-3}\\
&=s^T Y \tilde{a} + \tilde{a}^T \Gamma^{-1} \dot{\tilde{a}} - s^T K_D s \label{eq:ac-mrac-manipulator-controller-unknownparam-lyapunov-derivative},
\end{align}
where we used the manipulator dynamics \eqref{eq:ac-mrac-manipulator-dynamics} to rewrite \(H\ddot{q}\) in \eqref{eq:ac-mrac-manipulator-controller-unknownparam-lyapunov-derivative-1}, used \(\dot{H} - 2C\) is skew-symmetric in \eqref{eq:ac-mrac-manipulator-controller-unknownparam-lyapunov-derivative-2}, invoked the adapative control law \eqref{eq:ac-mrac-manipulator-controller-unknownparam} and reused \(Ya = H \ddot{q}_r + C \dot{q}_r + g(q)\) in \eqref{eq:ac-mrac-manipulator-controller-unknownparam-lyapunov-derivative-3}. The derivation above explains why the choice of the control law in \eqref{eq:ac-mrac-manipulator-controller-unknownparam} did not exactly follow its counterpart when the parameters are known: we need to use \(s^T Cs\) to cancel \(\frac{1}{2} s^T \dot{H} s\) in \eqref{eq:ac-mrac-manipulator-controller-unknownparam-lyapunov-derivative-2}.

We then wonder if we can design \(\dot{\tilde{a}}\) such that \(\dot{V}\) in \eqref{eq:ac-mrac-manipulator-controller-unknownparam-lyapunov-derivative} is negative semidefinie? This turns out to be straightforward with the adaptation law
\begin{equation}
\dot{\tilde{a}} = -\Gamma Y^T s,
\label{eq:ac-mrac-manipulator-controller-unknownparam-adaptation-law}
\end{equation}
to make \(s^T Y \tilde{a} + \tilde{a}^T \Gamma^{-1} \dot{\tilde{a}}\) vanish and so
\[
\dot{V} = - s^T K_D s \leq 0.
\]

We are not done yet. To show \(s\) converges to zero (which is implied by \(\dot{V}\) converges to zero), by Barbalat's stability certificate \ref{thm:BarbalatStability}, it suffices to show
\[
\ddot{V} = -2 s^T K_D \dot{s}
\]
is bounded. We already know \(s\) and \(\tilde{a}\) are bounded, due to the fact that \(V\) in \eqref{eq:ac-mrac-manipulator-controller-unknownparam-lyapunov} is bounded. Therefore, we only need to show \(\dot{s}\) is bounded. To do so, we plug the adapative control law \eqref{eq:ac-mrac-manipulator-controller-unknownparam} into the manipulator dynamics \eqref{eq:ac-mrac-manipulator-dynamics} and obtain
\[
H \dot{s} + (C + K_D) s = Y\tilde{a},
\]
which implies the boundedness of \(\dot{s}\) (note that \(H\) is uniformly positive definite, i.e., \(H \succeq \alpha I\) for some \(\alpha > 0\)). This concludes the analysis of the tracking convergence \(s \rightarrow 0\) as \(t \rightarrow \infty\).

\hypertarget{certainty-equivalent-adaptive-control}{%
\section{Certainty-Equivalent Adaptive Control}\label{certainty-equivalent-adaptive-control}}

\hypertarget{appendix-appendix}{%
\appendix}


\hypertarget{linear-system-theory}{%
\chapter{Linear System Theory}\label{linear-system-theory}}

Consider the linear time-invariant (LTI) system
\begin{equation}
\dot{x} = A x + B u, \quad y = C x + D u,
\label{eq:app-linear-system}
\end{equation}
where \(x \in \mathbb{R}^n\) the state, \(u \in \mathbb{R}^m\) the control, and \(A,B,C,D\) are constant matrices with proper sizes.

\begin{theorem}[Detectability]
\protect\hypertarget{thm:ltidetectable}{}\label{thm:ltidetectable}

The following are equivalent for the LTI system \eqref{eq:app-linear-system}:

\begin{enumerate}
\def\labelenumi{\arabic{enumi}.}
\item
  \((A,C)\) is detectable;
\item
  The matrix \(\begin{bmatrix} A - \lambda I \\ C \end{bmatrix}\) has full column rank for all \(\lambda \in \mathbb{C}\) such that \(\mathrm{Re}(\lambda) \geq 0\) ;
\item
  For all \(\lambda\) and \(x\) such that \(Ax = \lambda x\) and \(\mathrm{Re}(\lambda) \geq 0\), \(Cx \neq 0\);
\item
  There exists a matrix \(L\) such that \(A + LC\) is Hurwitz;
\item
  \((A^*,C^*)\) is stabilizable.
\end{enumerate}

\end{theorem}

\hypertarget{appconvex}{%
\chapter{Convex Analysis and Optimization}\label{appconvex}}

\hypertarget{the-kalman-yakubovich-lemma}{%
\chapter{The Kalman-Yakubovich Lemma}\label{the-kalman-yakubovich-lemma}}

\begin{lemma}[Kalman-Yakubovich]
\protect\hypertarget{lem:KalmanYakubovich}{}\label{lem:KalmanYakubovich}Consider a controllable linear time-invariant system
\[
\dot{x} = A x + b u \\
y = c^T x.
\]
The transfer function
\[
h(p) = c^T (p I - A)^{-1} b 
\]
is strictly positive real (SPR) if and only if there exist positive definite matrices \(P\) and \(Q\) such that
\[
A^T P + P A = - Q \\
Pb = c.
\]
\end{lemma}

\hypertarget{feedbacklinearization}{%
\chapter{Feedback Linearization}\label{feedbacklinearization}}

\hypertarget{slidingcontrol}{%
\chapter{Sliding Control}\label{slidingcontrol}}

  \bibliography{book.bib,packages.bib}

\end{document}
