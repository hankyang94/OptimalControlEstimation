% Options for packages loaded elsewhere
\PassOptionsToPackage{unicode}{hyperref}
\PassOptionsToPackage{hyphens}{url}
%
\documentclass[
]{book}
\usepackage{amsmath,amssymb}
\usepackage{iftex}
\ifPDFTeX
  \usepackage[T1]{fontenc}
  \usepackage[utf8]{inputenc}
  \usepackage{textcomp} % provide euro and other symbols
\else % if luatex or xetex
  \usepackage{unicode-math} % this also loads fontspec
  \defaultfontfeatures{Scale=MatchLowercase}
  \defaultfontfeatures[\rmfamily]{Ligatures=TeX,Scale=1}
\fi
\usepackage{lmodern}
\ifPDFTeX\else
  % xetex/luatex font selection
\fi
% Use upquote if available, for straight quotes in verbatim environments
\IfFileExists{upquote.sty}{\usepackage{upquote}}{}
\IfFileExists{microtype.sty}{% use microtype if available
  \usepackage[]{microtype}
  \UseMicrotypeSet[protrusion]{basicmath} % disable protrusion for tt fonts
}{}
\makeatletter
\@ifundefined{KOMAClassName}{% if non-KOMA class
  \IfFileExists{parskip.sty}{%
    \usepackage{parskip}
  }{% else
    \setlength{\parindent}{0pt}
    \setlength{\parskip}{6pt plus 2pt minus 1pt}}
}{% if KOMA class
  \KOMAoptions{parskip=half}}
\makeatother
\usepackage{xcolor}
\usepackage{longtable,booktabs,array}
\usepackage{calc} % for calculating minipage widths
% Correct order of tables after \paragraph or \subparagraph
\usepackage{etoolbox}
\makeatletter
\patchcmd\longtable{\par}{\if@noskipsec\mbox{}\fi\par}{}{}
\makeatother
% Allow footnotes in longtable head/foot
\IfFileExists{footnotehyper.sty}{\usepackage{footnotehyper}}{\usepackage{footnote}}
\makesavenoteenv{longtable}
\usepackage{graphicx}
\makeatletter
\def\maxwidth{\ifdim\Gin@nat@width>\linewidth\linewidth\else\Gin@nat@width\fi}
\def\maxheight{\ifdim\Gin@nat@height>\textheight\textheight\else\Gin@nat@height\fi}
\makeatother
% Scale images if necessary, so that they will not overflow the page
% margins by default, and it is still possible to overwrite the defaults
% using explicit options in \includegraphics[width, height, ...]{}
\setkeys{Gin}{width=\maxwidth,height=\maxheight,keepaspectratio}
% Set default figure placement to htbp
\makeatletter
\def\fps@figure{htbp}
\makeatother
\setlength{\emergencystretch}{3em} % prevent overfull lines
\providecommand{\tightlist}{%
  \setlength{\itemsep}{0pt}\setlength{\parskip}{0pt}}
\setcounter{secnumdepth}{5}
\usepackage{booktabs}
\usepackage{amsthm}
\makeatletter
\def\thm@space@setup{%
  \thm@preskip=8pt plus 2pt minus 4pt
  \thm@postskip=\thm@preskip
}
\makeatother
\ifLuaTeX
  \usepackage{selnolig}  % disable illegal ligatures
\fi
\usepackage[]{natbib}
\bibliographystyle{apalike}
\IfFileExists{bookmark.sty}{\usepackage{bookmark}}{\usepackage{hyperref}}
\IfFileExists{xurl.sty}{\usepackage{xurl}}{} % add URL line breaks if available
\urlstyle{same}
\hypersetup{
  pdftitle={Optimal Control and Estimation},
  pdfauthor={Heng Yang},
  hidelinks,
  pdfcreator={LaTeX via pandoc}}

\title{Optimal Control and Estimation}
\author{Heng Yang}
\date{2023-06-07}

\usepackage{amsthm}
\newtheorem{theorem}{Theorem}[chapter]
\newtheorem{lemma}{Lemma}[chapter]
\newtheorem{corollary}{Corollary}[chapter]
\newtheorem{proposition}{Proposition}[chapter]
\newtheorem{conjecture}{Conjecture}[chapter]
\theoremstyle{definition}
\newtheorem{definition}{Definition}[chapter]
\theoremstyle{definition}
\newtheorem{example}{Example}[chapter]
\theoremstyle{definition}
\newtheorem{exercise}{Exercise}[chapter]
\theoremstyle{definition}
\newtheorem{hypothesis}{Hypothesis}[chapter]
\theoremstyle{remark}
\newtheorem*{remark}{Remark}
\newtheorem*{solution}{Solution}
\begin{document}
\maketitle

{
\setcounter{tocdepth}{1}
\tableofcontents
}
\hypertarget{preface}{%
\chapter*{Preface}\label{preface}}
\addcontentsline{toc}{chapter}{Preface}

This is the textbook for Harvard ES/AM 158: Introduction to Optimal Control and Estimation. Information about the offerings of the class is listed below.

\hypertarget{fall}{%
\subsubsection*{2023 Fall}\label{fall}}
\addcontentsline{toc}{subsubsection}{2023 Fall}

\textbf{Time}: Mon/Wed 2:15 - 3:30pm

\textbf{Location}: Science and Engineering Complex, Room TBD

\textbf{Instructor}: \href{https://hankyang.seas.harvard.edu/}{Heng Yang}

\textbf{Teaching Fellow}: \href{https://scholar.harvard.edu/weiyuli/home}{Weiyu Li}

\href{https://docs.google.com/document/d/1q8_jB5dLx9jHOBi3DQ48Vv2E243ocGCGm_H0mJuOojM/edit?usp=sharing}{\textbf{Syllabus}}

\hypertarget{acknowledgment}{%
\subsubsection*{Acknowledgment}\label{acknowledgment}}
\addcontentsline{toc}{subsubsection}{Acknowledgment}

\hypertarget{formulation}{%
\chapter{The Optimal Control Formulation}\label{formulation}}

\hypertarget{the-basic-problem}{%
\section{The Basic Problem}\label{the-basic-problem}}

Consider a discrete-time dynamical system
\begin{equation}
x_{k+1} = f_k (x_k, u_k, w_k), \quad k =0,1,\dots,N-1
\label{eq:discrete-time-dynamics}
\end{equation}
where

\begin{itemize}
\item
  \(x_k \in \mathbb{X} \subseteq \mathbb{R}^n\) is the \emph{state} of the system,
\item
  \(u_k \in \mathbb{U} \subseteq \mathbb{R}^m\) is the \emph{control} we wish to design,
\item
  \(w_k \in \mathbb{W} \subseteq \mathbb{R}^p\) a random \emph{disturbance} or noise (e.g., due to unmodelled dynamics) which is described by a probability distribution \(P_k(\cdot \mid x_k, u_k)\) that may depend on \(x_k\) and \(u_k\) but not on prior disturbances \(w_0,\dots,w_{k-1}\),
\item
  \(k\) indexes the discrete time,
\item
  \(N\) denotes the horizon,
\item
  \(f_k\) models the transition function of the system (typically \(f_k \equiv f\) is time-invariant, especially for robotics systems; we use \(f_k\) here to keep full generality).
\end{itemize}

\begin{remark}[Deterministic v.s. Stochastic]
When \(w_k \equiv 0\) for all \(k\), we say the system \eqref{eq:discrete-time-dynamics} is \emph{deterministic}; otherwise we say the system is \emph{stochastic}. In the following we will deal with the stochastic case, but most of the methodology should carry over to the deterministic setup.
\end{remark}

We consider the class of \emph{controllers} (also called \emph{policies}) that consist of a sequence of functions
\[
\pi = \{ \mu_0,\dots,\mu_{N-1} \},
\]
where \(\mu_k (x_k) \in \mathbb{U}\) for all \(x_k\), i.e., \(\mu_k\) is a \emph{feedback} controller that maps the state to an admissible control. Given an initial state \(x_0\) and an admissible policy \(\pi\), the state \emph{trajectory} of the system is a sequence of random variables that evolve according to
\begin{equation}
x_{k+1} = f_k(x_k,\mu_k(x_k),w_k), \quad k=0,\dots,N-1
\label{eq:closed-loop-state-trajectory}
\end{equation}
where the randomness comes from the disturbance \(w_k\).

We assume the state-control trajectory \(\{u_k\}_{k=0}^{N-1}\) and \(\{x_k \}_{k=0}^{N}\) induce an \emph{additive cost}
\begin{equation}
g_N(x_N) + \sum_{k=0}^{N-1} g_k(x_k,u_k)
\label{eq:additive-cost}
\end{equation}
where \(g_k,k=0,\dots,N\) are some user-designed functions.

With \eqref{eq:closed-loop-state-trajectory} and \eqref{eq:additive-cost}, for any admissible policy \(\pi\), we denote its induced \emph{expected cost} with initial state \(x_0\) as
\[
J_\pi (x_0) = \mathbb{E} \left\{ g_N(x_N) + \sum_{k=0}^{N-1} g_k (x_k, \mu_k(x_k))  \right\},
\]
where the expectation is taken over the randomness of \(w_k\).

\hypertarget{literature}{%
\chapter{Literature}\label{literature}}

Here is a review of existing methods.

\hypertarget{methods}{%
\chapter{Methods}\label{methods}}

We describe our methods in this chapter.

Math can be added in body using usual syntax like this

\hypertarget{math-example}{%
\section{math example}\label{math-example}}

\(p\) is unknown but expected to be around 1/3. Standard error will be approximated

\[
SE = \sqrt(\frac{p(1-p)}{n}) \approx \sqrt{\frac{1/3 (1 - 1/3)} {300}} = 0.027
\]

You can also use math in footnotes like this\footnote{where we mention \(p = \frac{a}{b}\)}.

We will approximate standard error to 0.027\footnote{\(p\) is unknown but expected to be around 1/3. Standard error will be approximated

  \[
  SE = \sqrt(\frac{p(1-p)}{n}) \approx \sqrt{\frac{1/3 (1 - 1/3)} {300}} = 0.027
  \]}

\hypertarget{applications}{%
\chapter{Applications}\label{applications}}

Some \emph{significant} applications are demonstrated in this chapter.

\hypertarget{example-one}{%
\section{Example one}\label{example-one}}

\hypertarget{example-two}{%
\section{Example two}\label{example-two}}

\hypertarget{final-words}{%
\chapter{Final Words}\label{final-words}}

We have finished a nice book.

  \bibliography{book.bib,packages.bib}

\end{document}
